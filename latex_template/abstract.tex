\begin{abstract}
Epithelial sheets are active viscoelastic materials that form
specialized 3D structures suited to their physiological roles, such as
branched alveoli in the lungs, tubes in the kidney, and villi in the
intestine. The shape of these structures depends on active stresses
generated by the actomyosin cytoskeleton, active viscoelastic properties
of the epithelium, and hydraulics of the luminal fluid. How these active
stress and material properties are linked to give rise to epithelial
shape remains largely unknown. Here we developed a new experimental and
computational approach to probe active epithelial viscoelasticity and
then harness the resulting constitutive relation to sculpt epithelia of
controlled 3D shape. We developed a microfluidic setup to engineer 3D
epithelial tissues with controlled shape and pressure. In this setup, an
epithelial monolayer is grown on a porous surface with circular low
adhesion zones (footprint). On applying hydrostatic pressure, the
monolayer delaminates into a spherical cap (dome) from the circular
footprint. Through this approach, we subject MDCK epithelial cells to a
range of lumen pressures at different rates and hence probe the relation
between strain and tension in different regimes. Slow pressure changes
relative to the timescales of actin dynamics allow the tissue to
accommodate large strain variations. However, under sudden pressure
reductions, the tissue develops buckling patterns and folds with
different degrees of symmetry-breaking to store excess tissue area. This
behavior is well captured by a 3D computational model that incorporates
the turnover, viscoelasticity and contractility of the actomyosin
cortex. Informed by this model, we harness the active behavior of the
cell cortex to pattern epithelial folds by rationally directed buckling.
Our study establishes a new approach for engineering epithelial
morphogenetic events.
\end{abstract}

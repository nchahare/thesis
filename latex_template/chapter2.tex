
\hypertarget{morphogenesis}{%
	\chapter{Morphogenesis}\label{morphogenesis}}

During embryonic development, epithelia forms transient structures, such
as the neural tube, somites, and the precardiac epithelium, that serve
as progenitors for the development of more complex organs. Different
epithelia acquire diverse morphological forms and performs their
specific functions, such as the thyroid follicles, the kidney tubules,
the interconnected bile canaliculi and sinusoids in the liver, and the
complex branching structures found in the lung and salivary glands
(Gumbiner, B.M., 1992). Owing to its multifaceted regulation and
hierarchical organization, epithelial morphogenesis is a complex
phenomenon dependent on factors at multi spatial-temporal scales.

It can be fast at cellular level like the change in cell shape driven by
apical constrictions, which is required for epithelial remodeling during
tube formation of ventral furrow cells in Drosophila gastrulation
(Miller CJ, Davidson LA. 2013). Or it could be a slower
self-organization at embryo level like a cluster of dissociated mouse
embryonic stem cells (ESCs) cultured in vitro spontaneously form an
optic cup, exhibiting all layers of the neural retina, when cultured in
appropriate medium (Eiraku et al., 2011; Bedzhov, I. \& Zernicka-Goetz,
M. 2014). This structure underwent similar changes to the in vivo tissue
like invaginating to form the characteristic morphology of the optical
cup without external scaffolding or original growth environment.

\begin{quote}
	Figure: slow vs fast process; programmed vs self-organizing structure;
	function and form logic!
\end{quote}

At the end of the day, all cells come from cells (`omnis cellula e
cellula') (Virchow R.L.K., 1858), all tissues come from cells that
contain essentially the same genetic information. Nonetheless, every
tissue exhibits a distinct architecture and function. One could ask many
questions from here, as how form-function work in synchrony? How
organization is triggered physically? Reductionists would ask whether
function follows form, or it is other way around.

As per the twentieth century architecture principle of ``Form Follows
Function''; where the organization of a structure should be based upon
its intended function. In developmental biology there are many examples
of indicating that the same principle is at work in self-assembling
systems like intestinal organoids, cancerous spheroids, and functional
kidney tissues (Gjorevski, N, et al.~2016; Ishiguro, T, et al.~2017;
Morizane, R. and Bonventre, J.V., 2017). Each emerging out of a set of
cells in appropriate environment changing and adapting itself to perform
the biological function. However, exactly the opposite design principle
is at work in numerous in vitro experiments with controlled cellular
environment; illustrating geometric constraints drives biological
function. For instance, in a micropatterned collagen scaffold (with
structures of intestine) a human small intestinal epithelium was
generated that replicates key features of the in vivo small intestine: a
crypt-villus architecture with appropriate cell-lineage
compartmentalization and an accessible luminal surface (Wang, Y et al.,
2017). Or cell reprogramming like in case of fibroblasts turning into
induced neurons when supported by specific substrate topography
(Kulangara et al.~2014).

\begin{quote}
	Figure: images from D'arcy thompson book and connecting it to the new
	work of forces.
\end{quote}

One could easily reach a conclusion that there are more things involved
in understanding dialectics of form and function. This was a subject of
D'Arcy Wentworth Thompson's classical text ``On Growth and Form''
(Thompson, 1917). Thompson tries to explore biological forms during
development and across evolution with considering geometric and physical
constraints. (\emph{Here talk more about specific examples from the
	book}) After more than 100 years of its publishing we can answer more
specific questions about shape and function using advances in
bioengineering and microscopy.

\begin{quote}
	Structure without function is a corpse, function without structure is a
	ghost (Wainwright, S.A., 1988.)
\end{quote}

It is quite apparent after reading till here that there is a specter is
haunting this subject---the specter of force. In last couple of decades,
there has been a resurgence of interest in physical forces as regulator
of development, homeostasis, and disease (Ingber, D. 2005; Barnes, J.M.
et al., 2017). This has led researchers across the disciplines to
examine the physical mechanisms of tissue formation and its regulation.
Unravelling mechanism of Thompson's mysteriously generated `Diagram of
the forces' which governs biological processes (Thompson, 1917; Heer,
N.C. and Martin, A.C., 2017).

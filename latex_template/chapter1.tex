The center focus of this thesis is epithelial monolayers. From the
materialist perspective, these monolayers are endlessly fascinating.
They are shape changing, self healing, continuously deform or jam
depending on the requirement. They are the simplest system conceptually
to understand the wider world of morphogenesis. This introduction is a
primer to all the topics relevant to the thesis. First I will give a
brief introduction to the epithelial tissue itself and its key
components along side a snippet of its role in disease and development.
Then I will give a summary of morphogenesis and how we can think of
epithelia as an active material and what are the historical ways of
modelling it. Finally, I will conclude with the emerging field of bottom
up morphogenesis, where researchers are reconstructing the biological
systems from scratch.

\hypertarget{epithelial-tissue}{%
	\chapter{Epithelial tissue}\label{epithelial-tissue}}

Epithelial tissues are cell sheets with strong intercellular bonds that
form physical barriers for major organs such as the lungs, skin, and
intestine. It protects the organs from external physical, chemical, and
microbial onslaughts. Besides protection, the main functions of
epithelial cells include secretion, selective absorption, transcellular
transport, and detection of sensation (Powell, D.W., 1981). It also
plays a key role in developmental stages by supporting growth and
driving critical shape changes.

Epithelial cells are polarized, i.e., their apical side, faces the lumen
of the organ, differs in shape and composition from the basolateral
side. Its polar organization is reflected in the vectoral functions like
transporting epithelia such as those of the renal tubule, absorptive
epithelia of the intestine, and secretory epithelial cells like
hepatocytes, which are typical examples of epithelia that create and
maintain concentration gradients between the separated compartments
(Simons, K. and Fuller, S.D., 1985). In addition, polarized epithelia
guide the developmental process by determining the fate of cells (Kim,
E.J.Y., Korotkevich, E. and Hiiragi, T., 2018).

Epithelial cells have different shapes and may be arranged in single or
multiple layers. They are usually classified according to two features:
the number of cell layers and the shape of the cells. Simple epithelia
are single-cell layers where all the cells contact the underlying basal
lamina and have an apical free surface. The shape of the cells can be
flat (wider than high), cuboidal (as wide as high), or columnar (higher
than wide). However, stratified epithelium contains two or more layers
of cells.

\begin{quote}
	Figure: Art work of quilling and then showing different epithelial types
	then showing apical basal polarity of epithelial and monolayer.
\end{quote}

Epithelial function primarily depends on the tissue's structure and
microenvironment. In essence, it can be described completely in three
parts: first, cell structure; second, cell-substrate connection; and
lastly, microenvironment.

In general, cell structure helps cells maintain their shape along with
providing mechanical support to perform vital functions like division
and migration. This structure is known as the cell cytoskeleton. It
includes different components, playing various roles together.
Eukaryotic cells are constructed out of filamentous proteins to support
the cell and its cytoplasmic constituents. There are three major
filaments, which differ in size and protein content. Microtubules are
the largest type of filament of the protein tubulin, with a diameter of
about 25 nm. Actin filaments are the smallest type, with a diameter of
only about 6 nm. Finally, intermediate filaments are medium-sized, with
a diameter of about 10 nm. Unlike actin filaments and microtubules,
intermediate filaments are constructed from several different subunit
proteins. These filaments dynamically alter themselves in reaction to
signals from microenvironments and cell networks (Alberts, B., et al.,
2013; Fletcher, D.A., and Mullins, R.D., 2010s). Mechanically, actin
filaments are stiffer than microtubules in extension, but they rupture
at lower extension. It is also reported that the intermediate filaments
exhibit an intermediate extensional stiffness at lower extensions, but
that the intermediate filament can sustain much larger extensions than
the other two types of filaments while exhibiting a nonlinear stiffening
response (Janmey et al., 1991; Mofrad, M.R., 2009).

In the case of epithelial layers, the actin cytoskeleton and
intercellular junctions make cell-cell contacts stronger and provide
integrity (Braga V. 2016). The perfect example of these tissue-level
structures can be seen in wound healing assays: cells surrounding the
wound create a ring of actin to close it (Brugués, A., et al., 2014).
One must keep in mind that these structures tend to self-organize as
well. It can be seen when cells are confined in a specific shape, like
in the case of circular islands of epithelial cells that showed radial
patterns in the actin organization (Jalal S., et al., 2019).

Multiple membrane molecules can mediate adhesion between cells. One of
these are cadherins, critical for epithelial cell cohesion through the
formation of adherens junctions. In these junctions, cadherins are
coupled to the cell cytoskeleton enabling force transmission between
cells. It is finely regulated by both internal and external mechanisms.
Desmosomes are another type of intercellular junction. They are coupled
with intermediate filaments, and the resulting supracellular network
confers mechanical resilience on cell layers (Hatzfeld, M., Keil, R., \&
Magin, T. M. 2017; Latorre, E., et al., 2018). Tight junctions (TJ)
perform a barrier function and enable the transport of ions across
epithelial layers to be actively regulated. This plays an important role
in the control of fluid pressure in tissues. Together, adherens
junctions, desmosomes, and tight junctions are the major mediators of
epithelial cell--cell adhesion, and their regulation enables emergent
behaviors in cell sheets that are not observed in single-cell systems
(Trepat, X., and Sahai, E., 2018; Ladoux, B., and Mège, R.M., 2017).

\begin{quote}
	Figure: all the cytoskeleton and cell junctions and their mechanics
\end{quote}

Extracellular matrix (ECM) is the cell environment or substrate to which
cells adhere; it is also known as the matrix or cellular
microenvironment. ECM serves many functions: it endows a tissue with
strength and thereby maintains its shape; it serves as a biologically
active scaffolding on which cells can migrate or adhere; it helps to
regulate the phenotype of the cells; it serves as an anchor for many
substances, including growth factors, proteases, and inhibitors of such;
and finally, it provides an aqueous environment for the diffusion of
nutrients, ions, hormones, and metabolites between the cell and the
capillary network. On top of that, it is subjected to mechanical forces
such as blood flow in endothelia, air flow in respiratory epithelia, or
hydrostatic pressure in the mammary gland and bladder (Roca-Cusachs, P.,
Conte, V., \& Trepat, X. 2017; Humphrey, J. D., et al., 2015; Waters, C.
M., Roan, E., \& Navajas, D., 2012; Paszek, M. J., \& Weaver, V. M.,
2004; Bross, S., et al., 2003). It is shown that the ECM regulates cell
shape, orientation, movement, and overall function in response to these
forces.

Cells and ECM have a symbiotic relationship with each other from
signaling cues to various sensors on the cell surface. These cues are
primarily sensed using integrins and focal adhesion complexes in
cell-substrate adhesion (Kechagia, J.Z., Ivaska, J. and Roca-Cusachs,
P., 2019). Thus, triggering complex molecular processes that are
required to maintain homeostasis and strongly affect processes in
development or tumorigenesis (DuFort, C. C., Paszek, M. J., \& Weaver,
V. M. 2011; Northey, J. J., Przybyla, L., \& Weaver, V. M. 2017).

ECM is a fibrous network of proteins; from a mechanical perspective, the
three primary structural constituents of the ECM are typically collagen
(the most abundant protein in the body), elastin (the most elastic and
chemically stable protein), and proteoglycans (which often sequester
significant water as well as growth factors, proteases, etc.). Due to
its water content, the deformation of ECM can produce cracks in
epithelial layers. ECM acts as a poroelastic material, soaking up water
upon stretching (like a sponge) and releasing it under compression,
causing a hydraulic fracture effect (Casares, L., et al., 2015).
Moreover, collagen remodels itself under the influence of cells aiding
in migration or under stress (Shields, M. A., et al., 2012; Humphrey, J.
D., 2003). Like most cytoskeletal proteins, most extracellular
components turnover continuously, albeit some very slowly. For example,
collagen in the peridontal ligament appears to have a half-life of a few
days, whereas that in the vasculature may have a normal half-life of
several months. In response to altered loads, disease, or injury,
however, the rates of synthesis and degradation of collagen can increase
many folds to have a rapid response (Humphrey J. D., 2003).

Epithelial integrity and homeostasis are of central importance to
survival, and mechanisms have evolved to ensure these processes are
maintained during growth and in response to damage (Macara, I.G., et
al., 2014). For example, epithelial cells have one of the fastest
turnover rates in the body. The entire gut cell lining turns over in
3--4 days. This turnover implies constant cell division and death. The
excessive rate of division and death may give rise to tumors. It is
known that 90\% of cancers emerge in simple epithelia (Torras, N. et
al.~2018; Eisenhoffer, G.T., and Rosenblatt, J., 2013). Not only this,
but it could easily disrupt the barrier function, as no gaps should
emerge around dying or dividing cells. There is a range of distinct
disease states that all have the effect of compromising epithelial
and/or endothelial barrier function.

If the fluid compartmentalization goes awry, it has profound
implications for epithelial and stromal homeostasis, fluid and/or
electrolyte balance, the generation of inflammatory states, and even the
tumor microenvironment. Several bacterial toxins are known to target
junctions and cause changes in the TJ protein ZO1, resulting in
compromised barrier function and pathologies such as diarrhea and
colitis (Fasano, A. et al., 1991). Cancer cells tend to spread and
disperse metastatically by using their very high rate of cell motility
and a diminished sense of cell adhesion. This elimination and/or
reduction of TJ barriers in cancer is essential to allowing metastatic
cells to break into and out of blood vessels. The leaky barrier also
allows a growing epithelial tumor to access luminal fluids as an
additional source of nutrients (Mullin, J.M., et al., 2005).

Epithelia undergo changes in shape with deformation and reorganization
from the embryonic to the adult stage. Unsurprisingly, any improper
function would lead to damage and disorder. Like in the case of
epithelial--mesenchymal transition (EMT), a developmental process when
epithelial cells gradually transform into mesenchymal-like cells by
losing their epithelial functionality. EMT is involved in the
pathogenesis of numerous lung diseases ranging from developmental
disorders to fibrotic tissue remodeling to lung cancer. Another example
is bronchopulmonary dysplasia (BPD), a chronic lung disease that occurs
in very premature infants and is characterized by impaired
alveologenesis and vascular development. BPD develops because of injury
or infection in a very immature lung (Bartis, D. et al., 2014). Give
better and more prominent examples then to conclude the chapter with
telling what follows.

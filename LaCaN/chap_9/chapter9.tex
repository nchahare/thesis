
\hypertarget{conclusions}{%
	\section{Conclusions}\label{conclusions}}

The main conclusions of this study are as follows:

\begin{enumerate}
	\def\labelenumi{\arabic{enumi}.}
	\item
	We have developed a microfluidics-based system for generating 3D
	epithelia using micropatterning technique PRIMO to create a
	non-adhesive region from which epithelial monolayers can detach and
	inflate into a dome.
	\item
	Using the MOLI technique, we probed the mechanics of epithelial domes
	subjected to constant pressure and found that the dome reaches a
	steady state after five minutes of applying pressure. These
	experiments revealed a non-monotonic tension-strain response due to
	geometric constraints.
	\item
	The constitutive response of the epithelial tissue showed that the
	domes exhibit an initial increase in tension with strain, tending to a
	tensional plateau at high strains, consistent with earlier studies
	demonstrating superelastic behavior in epithelia.
	\item
	Dynamic material testing of domes with varying inflation/deflation
	rates demonstrated that the active viscoelasticity of the cortical
	network directs epithelial rheology.
	\item
	We developed a complementary model, developed by Adam Ouzeri, to
	understand the different timescales involved in the tissue stretching
	process.
	\item
	The epithelial tissue behaves like an active viscoelastic fluid,
	exhibiting hyperelastic behavior at short timescales and viscoelastic
	behavior at slower scales.
	\item
	Rapid deflation to -50 Pa, faster than the remodeling timescales,
	leads to buckling instability.
	\item
	Buckling occurs at multiple scales, from the subcellular level to the
	tissue level, with differing characteristic lengths of the folds. The
	shortest folds occur at the membrane cortex level, and the longest at
	the tissue scale.
	\item
	Different sized spherical domes create different patterns. Smaller
	domes buckle into a fold along their periphery, while larger ones tend
	to create a network of folds in the center.
	\item
	Folds can be programmed by controlling the shape of the dome.
	Elliptical domes produce a fold along the major axis, while triangular
	domes produce a Y-shaped network in the middle.
\end{enumerate}

\hypertarget{future-perspectives}{%
	\section{Future Perspectives}\label{future-perspectives}}

The experiments and theory presented in this thesis focus on the active
viscoelasticity of tissues and the generation of folds using buckling
instability. However, this experimental setup has implications for
several projects within our research group.

For example, the current experiments only examined short timescales
(\textless10-30 minutes) and focused solely on the actin cytoskeleton.
Investigating the role of other cytoskeletal components, such as
intermediate filaments, would be of great interest. Past studies have
shown that intermediate filaments are critical in tissue re-stiffening.
My colleague, Tom Golde, is currently using MOLI to study intermediate
filament networks.

In addition, we are also utilizing the MOLI device to study a variety of
different systems, including stem-cells, cancerous tissues, and
organoids. This could enable us to investigate the interplay between
geometry, pressure, and cell fate. The inverted cell culture method we
use also allows for high-resolution imaging. Two-channel system provides
a conducive environment for maintaining complex culture conditions as
well as allow for co-culture possibilities.

For the mechanobiology community, our setup is particularly intriguing
because, when the domes are stretched beyond 100\%, the nucleus becomes
compressed, which triggers various mechanotransduction pathways. We
could easily examine the role of the nucleus and different
mechanosensitive proteins when subjected to deformation. Additionally,
we could use pharmacological treatments to alter tissue tension and
understand the specific molecular pathways involved in maintaining
tissue shape.

At the molecular scale, we could also investigate focal adhesions during
delamination. Our experimental system delaminates tissue from the
substrate, which presents an opportunity to study cell-substrate
adhesion using protein patterning and live-cell imaging with focal
adhesion markers. Furthermore, experiments conducted over longer
timescales could allow us to explore the mechanics of cell-cell
junctions and cellular rearrangements in response to prolonged
stretching.

I must note the tissue hydraulic aspect, which has been largely
overlooked in this thesis. Specifically, we have not extensively
explored the trans-epithelial flow due to its negligible effect on our
timescale. However, recent studies have demonstrated that the epithelial
tissue can function as an active mechano-biological pump, generating its
own pressure gradient over longer timescales. Thus, it would be
worthwhile to investigate the role of fluid transport under controlled
pressure in our microfluidic system.

Through my work with this system, I have identified numerous promising
directions for future research. One particularly intriguing project, led
by Thomas Wilson, involves the implementation of concepts such as shape
changing, self-healing, and flexible epithelia in the creation of
biohybrid devices. Our initial approach will involve the construction of
a microfluidic chip, where the channels are composed of epithelial
tissues that can be manipulated using optogenetic tools to open or close
specific segments, akin to valves. This endeavor will enable us to
generate novel synthetic epithelial tissue systems and develop a more
comprehensive understanding of the underlying physical principles
driving morphogenesis.

\section{Device Fabrication}

\subsection{Main Block Fabrication}

\begin{enumerate}
	  \setlength\itemsep{-0.1em}
	\item Place the empty 3D printed mold on hot plate at 100$^{\circ}$C for 1 hour.
	\item Mix around 20 ml of PDMS (Sylgard kit) at a 1:9 ratio. 
	\item Pour PDMS into a $50 \, \text{ml}$ centrifuge tube, and weight out another of equal mass of water.
	\item Centrifuge for 3 mins at $900 \, \text{rpm}$. 
	\item Fill the mold with $12 \, \text{ml}$ of PDMS.
	\item Place the mold with PDMS into a desiccator for 1 hour or until there are no bubbles remaining.
	\item Then place the mold on a hot plate at 100$^{\circ}$C for 30 mins, until it is polymerised and solid to touch with tweezers. 
	\item Carefully cut the PDMS block out from the mold and place it into a dish with the pattern side up.
	\item Repeat Steps 5 to 8 to create more blocks and then with any remaining PDMS, fill the mold with a small amount and set it on the hot plate 100$^{\circ}$C, so as to keep the mold clean.
	\item Punch four holes in the devices with the $1.5 \, \text{mm}$ biopsy punch.
	\item Cut out the individual devices with the scalpel.
	\item Store devices with the pattern facing up in a dish.
\end{enumerate}

\subsection{Membrane Fabrication}

\begin{enumerate}
	 \setlength\itemsep{-0.1em}
	\item Meanwhile, prepare around $10 \, \text{ml}$ of PDMS at a ratio of 2:8. Centrifuge as in Steps 3 and 4 in the previous section.
	\item Using a syringe, measure $4 \, \text{ml}$ of PDMS into a large petri dish.
	\item Use the nitrogen gun to remove bubbles and evenly spread the PDMS across the dish, whilst not letting it touch the sides.
	\item Leave the dish on a level surface at room temperature for at least 2 hours to allow the PDMS to coat evenly.
	\item Place in the oven at 80$^{\circ}$C for an hour (or 60$^{\circ}$C for 2 hours).
	\item Follow section 1.2 to cut the membranes.
\end{enumerate}

\subsubsection{Silhouette Cutting Method}

\begin{enumerate}
		 \setlength\itemsep{-0.1em}
	\setcounter{enumi}{6}
	\item Cut the biggest rectangle possible in the PDMS membrane with scalpel.
	\item Use a bit of $90\%$ ethanol to peel off the membrane from the dish.
	\item Place the membrane on the silhouette cutting mat. Make sure to have membrane completely dry before placing it.
	\item Press the membrane against the mat to get rid of all the bubbles and making adhesion with the mat more secure.
	\item Load the cutting mat in the Silhouette machine.
	\item Open cutting pattern in the silhouette software.
	\item Select the cut settings:
	\begin{itemize}
			 \setlength\itemsep{-0.1em}
		\item \textit{Material = Vinyl}
		\item \textit{Depth = 10}
		\item \textit{Speed = 5}
	\end{itemize}
	\item Make sure the PDMS is at a correct location corresponding to the drawings.
	\item Press `\textit{Send to silhouette machine}'.
	\item During the cutting process, be aware of pieces of PDMS delaminating and getting stuck in the machine.
	\item After the successful cutting, unload the cutting mat.
	\item Remove all the small PDMS and holes before then peeling off the actual layers and storing them in a dish.
\end{enumerate}

\section{Plasma Bonding}

\subsection{First Bonding}

\begin{enumerate}
			 \setlength\itemsep{-0.1em}
	\item Items needed in the clean room: \begin{itemize}
			 \setlength\itemsep{-0.1em}
		\item PDMS blocks and membranes
		\item Glass bottom dishes
		\item Scalpel
		\item Tweezers
		\item Cutting mat
	\end{itemize}
	\item Have a hot plate next to the plasma cleaner at 100$^{\circ}$C with cloth on top.
	\item Plasma clean a large dish with the top blocks and the middle layers. Use the high setting for 30 seconds.
	\item Combine the block and membrane (so as to create an enclosed channel with a central hole). Make sure to line up holes and apply a lot of pressure to the PDMS.
	\item Flip over the blocks (membrane side up) in the dish, and place on the hot plate.
	\item Place the bottom channel layers into the lids of the glass bottom dishes, with two membranes per lid.
	\item Plasma clean the layers and glass bottom dishes the same as Step 4.
	\item Holding each dish at arms length, spray it with ethanol so that it gets very lightly coated, and then attach a membrane to the centre of the dish. Apply some pressure and ensure that there are no bubbles but do not apply to much force as the glass bottoms of the dishes are fragile. 
	\item Put the lids onto the dishes and place them onto the hot plate.
	\item Once plasma area is cleaned, place the samples into the oven at 80$^{\circ}$C. Leave for a minimum of 2 hours.
\end{enumerate}

\subsection{Second Bonding}

\begin{enumerate}
 \setlength\itemsep{-0.1em}
	\item Cut any protruding parts of the membrane from the bonded PDMS blocks, and then cut the corners off to create an octagonal shape.
	\item Cut $400 \, \text{nm}$ porous membrane disks into small pieces of around $4 \, \text{mm} \: \times \: 4 \, \text{mm}$ (9 pieces per sheet). Store these between layers of a cloth in a large petri dish.
	\item Mix around $20 \, \text{ml}$ of PDMS and put small amount (<2ml) into a petri dish.
	\item On the underside of the glass bottom dishes, mark the edges of the bonded channel with a pen.
	\item Return to clean room with devices, membranes and unpolymerised PDMS.
	\item Have a hot plate at 100$^{\circ}$C.
	\item Hold a piece of paper with tweezers, slightly dip one edge of the paper into the wet PDMS and remove excess by placing it on a cloth. 
	\item Gently, paint the wet PDMS onto the middle layer around the hole with paper. Be cautious of using too much PDMS might result in blockage on the hole.
	\item Place the porous membranes over the central hole of the PDMS block and press down the edges gently. 
	\item Similar to the first bonding, place the blocks into the glass bottom dish lids with two per lid. 
	\item Plasma clean the blocks and glass bottom dishes, following the same specifications as Step 4 in Section 2.1.
	\item Bond the blocks and dishes by placing the blocks porous membrane side up on your finger and align the channel using the pen marks as a guide. Ensure that the holes are lined up and the two enclosed channels are perpendicular. Press the PDMS block on the cloth to ensure attachment. 
	\item Put lids onto the dishes and place them onto the hot plate.
	\item Once plasma area is cleaned, return with the samples and place them into the oven at 80$^{\circ}$C. 
	\item After 30 mins, use a $20 \, \text{ml}$ syringe to pour PDMS around the base of the devices for a better seal, covering the remainder of the glass base.
	\item Return the devices to the oven and leave for a minimum of 2 hours.
\end{enumerate}

\section{PRIMO}

\subsection{PRIMO Preparation}

\begin{enumerate}
	 \setlength\itemsep{-0.1em}
	\item Wash the devices with ethanol. Pipette $80 \, \text{\textmu l}$ of $70\%$ ethanol thought each channel, ensuring that it only emerges for the opposite hole, and that there are no bubbles in the channels. Aspirate off the excess.
	\item Repeat this with MilliQ water or PBS.
	\item Pour a few drops of MilliQ water around the edge of the device to reduce evaporation.
	\item Fill each channels with $80 \, \text{\textmu l}$ of PLL.
	\item Leave at room temperature for 1 hour.
	\item After 50 mins, create SVAPEG solution (will require $80 \, \text{\textmu l}$ per device so scale accordingly):
	\begin{itemize}
		 \setlength\itemsep{-0.1em}
		\item Accurately measure around $20 \, \text{mg}$ of SVAPEG powder.
		\item Add $20 \times$ the amount of HEPES buffer solution to the powder ($400 \, \text{\textmu l}$ for $20 \, \text{mg}$ of powder). 
		\item Mix well, ensuring all powder is included. Do this immediately before using.
		\item Using a desiccator, store the SVAPEG powder under argon, sealed with para-film and in a vacuum bag.
	\end{itemize}
	\item Pipette $40 \, \text{\textmu l}$ of HEPES buffer solution into each channel and aspirate the excess.
	\item Soon after mixing the SVAPEG solution, fill each channel with $40 \, \text{\textmu l}$ of it.
	\item Leave for 30 mins at room temperature or overnight at $4^{\circ}$C, with enough MilliQ water around them to prevent them becoming dry.
\end{enumerate}

\subsection{PRIMO Protocol}

\begin{enumerate}
	 \setlength\itemsep{-0.1em}
	\item Prepare the fibronectin solution (will require $40 \, \text{\textmu l}$ per device):
	\begin{itemize}
		 \setlength\itemsep{-0.1em}
		\item Combine $20 \, \text{\textmu l}$ of fibrinogen [FGN] with $980 \, \text{\textmu l}$ of of PBS.
		\item Filter the solution with $0.22 \, \text{\textmu m}$ filter.
		\item Take $600 \, \text{\textmu l}$ of the mixture and add $60 \, \text{\textmu l}$ of fibronectin [FN]. Discard leftovers.
		\item Store in the ice.
	\end{itemize}
	\item Add $40 \, \text{\textmu l}$ of photo-inhibitor to the lower channel of the devices. Do this in batches of three or four.
	\item Preform PRIMO:
	\begin{itemize}
		 \setlength\itemsep{-0.1em}
		\item Use the PRIMO enabled microscope.
		\item Focus on porous membrane, which will be when it appears darkest.
		\item Find boundary of the $1.2 \, \text{mm}$ hole and match $1200 \, \text{\textmu m}$ circle.
		\item Select `\textit{Scan}' and then `\textit{Lock}'.
		\item Select `\textit{\textmu Pattern}' and load template.
		\item Select \textit{Lines = 4} and \textit{Dose = 500}.
		\item Set Spacing to the negative of the `\textit{Find Size Height}' ($-1223 \, \text{\textmu m}$), but adjust to get the best pattern match.
		\item Turn off both `\textit{Eye}' Icons on the left side.
		\item Select `Forward' in lower-right corner of screen and wait for the process to complete, which should be around 17 mins.
	\end{itemize}
	\item After PRIMO is complete, pipette $80 \, \text{\textmu l}$ of PBS into each channel and aspirate excess.
	\item Then pipette $40 \, \text{\textmu l}$ of FN solution (from Step 1) into the lower channel. Do this in batches of three or four.
	\item After exactly 5 mins (important to be accurate), wash all channels with PBS twice.
	\item Place a few drops of MiliQ water around the devices and store in fridge at $4^{\circ}$C. Note that cells must be seeded with 2 days of PRIMO completion. 
\end{enumerate}

\section{Cell Seeding}

\begin{enumerate}
			 \setlength\itemsep{-0.1em}
	\item Warm media and trypsin in the cell culture bath.
	\item Treat all devices with UV for 15 mins using the fume hood (`\textit{Disinfect}' $\rightarrow$ `\textit{UV}'). Ensure all lids are opened. 
	\item Detach cells from the flask using $3 \, \text{ml}$ trypsin, as normally done in cell passing.
	\item While cells are in the oven detaching, get two $15 \, \text{ml}$ vials and fill one with media.
	\item Aspirate the top of the devices and fill all channels with media (from vial). Aspirate the excess.
	\item After removing cells from the oven, mix with $7 \, \text{ml}$ of media. 
	\item Using a Neubauer Chamber, count the number of cells in the sample.
	%\begin{itemize}
	%\item Pipette around $10 \, \text{\textmu l}$ of culture into the chamber, until it has filled uniformly. Ensure sample is well mixed before. 
	%\item Count the number of cells in one for the long rectangles shown below. Count cells touching the upper and left limits, but not those touching the lower or right limits.
	%\begin{figure*}[h!]
	%\centering
	%\includegraphics[height=0.2\textwidth]{Neubauer.png}
	%\end{figure*}
	%\item Repeat with multiple rectangles and take an average to get a more accurate value.
	%\item Multiply the counted number of cells by $10^4$ to get the total number of cells in the culture per millilitre.
	%\item Clean afterwards with bleach and ethanol. Dry with the air gun.
	%\end{itemize}
	\item Meanwhile, fill the other $15 \, \text{ml}$ vial with the rest of the cell culture and centrifuge for 3.5 mins at $1000 \, \text{rpm}$.
	\item Aspirate away the media from the vial, leaving only the cells at the bottom.
	\item Add new media, such that the sample is diluted to 25-30 million cells per ml.
	\item Pipette $35 \, \text{ml}$ of culture into the lower channel of each device. Always mix the cells in the vial with reverse pipetting between samples and quickly aspirate off the excess. Put a few drop of media on top and around each device.
	\item Flip tray over and place in the incubator.
	\item After 1 hour, flip the devices back round and aspirate any media off the device tops.
	\item Quickly pipette $200 \, \text{\textmu l}$ of media into the device lower channels to remove unattached cells, while aspirating off the excess media.
	\item Put a few drops of fresh media on top of the devices to slow evaporation.
	\item Using a microscope, check the devices. Use the CY5 channel to check for a pattern, and the GFP channel for a uniform layer of cells.
	\item Rinse devices again with media and place a large droplet on top of the devices such that when the lids are replaced, a `column' of media is formed.
	\item Place in the incubator again (the correct way up) and leave overnight.
\end{enumerate}

\section{Experiment}

\begin{enumerate}
 \setlength\itemsep{-0.1em}
	\item Warm up two reservoirs of CO$_2$ media (one of around $35 \, \text{ml}$ and one of around $10 \, \text{ml}$) for around 30 mins.
	\item Then place the reservoirs in a desiccator for 30 mins (with lids unscrewed).
	\item Clean tubing with PBS.
	\item Connect the translation to both power and the computer. Place the $35 \, \text{ml}$ of media in the stage holder.
	\item Connect vacuum and open to  $200 \, \text{psi}$. Tape tubes into place around the microscope.
	\item Open stage control software \textit{ZABER}, and move stage to where the level of the media is the same as that of the devices.
	\item Move condenser as high as possible and connect pressure stage tubing to one of the upper-channel device outlets. Use the vacuum to ensure the full tube is filled with media before connecting, as it is important that there are no bubbles in the tubing or device.
	\item Connect the vacuum T-device to the other upper-channel device outlet.
	\item Slowly lift the media reservoir out of the stage until domes start to appear in patterned areas. Return reservoir to stage.
	\item Using stage software, re-form domes and identify the best one(s) for imaging.
	\item Perform the experiment of choice.
	\item After experiment, disconnect everything and clean tubing with ethanol and PBS.
\end{enumerate}

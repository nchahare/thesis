\hypertarget{introduction}{%
\section{Introduction}\label{introduction2}}

Our motivation in this thesis is two fold: first to create three
dimensional epithelia with controlled pressure, and second to study the
material response of the tissue to different regimes of tension. The
first goal has been achieved through fabrication of monolayer inflator
(MOLI) device. We are able to form epithelia domes where cells are
stretched for more than 100\% of area strain.

Because of the physiological functions of epithelia, tissue has to
undergo deformation over a wide range of timescales and magnitudes. Same
for pressure, it is recorded that the luminal pressure in blastocysts
doubles over its development leading changes in cortical tension and
strain. MDCK dome system is a perfect system where we get to see the
interplay between cell strain, tension, pressure. Latorre et al observed
wide range of the pressure throughout evolution of the dome; and cells
go through range of deformation including displaying active-superelastic
behavior. However, the control is applied only on the footprint of the
domes. In this chapter, we would use the MOLI system to actually subject
tissues to range of strain and tensions.

\hypertarget{measurement-of-dome-mechanics}{%
\section{Measurement of dome
mechanics}\label{measurement-of-dome-mechanics}}

To measure kinematics of the domes, we analyzed midsection of the domes
assuming symmetry of spherical caps. We measure the height \(h\) and
base radius \(a\). This allows us to calculate radius of curvature \(R\)
using trigonometry. \[ R = \frac{h^2 + a^2}{2h} \] This also allows us
to compute tension \(\sigma\) given by Laplace's law.
\[\sigma = \frac{\Delta PR }{2}\] For dome strain, we use areal strain
measure which is strain computed on basis of surface area. We compare
dome surface area to the area of footprint.
\[ \epsilon = \frac{A_{dome } - A_{footprint}}{A_{footprint}} \]
\[ \epsilon = \frac{\pi(h^2 + a^2) - \pi a^2}{\pi a^2} = \frac{h^2}{a^2}\]
Using line scan method of imagining domes for time-lapse, we get large
amount frames for analysis of height and radius of curvature. Thus, we
use kymographs to of the top section of the dome. Kymographs are the
images with cross section of dome top with respect to time. Using image
processing MATLAB code, we get the location of maximum intensity value
for particular time in the graph, to obtain the time evolution height of
dome. Same is done for the base radius too. The kymograph of base radius
allows us to keep track of the delamination. Because delamination would
change the value of strain.

\hypertarget{epithelial-domes-at-constant-pressure}{%
\section{Epithelial domes at constant
pressure}\label{epithelial-domes-at-constant-pressure}}

First we decided to study behavior of domes subjected to the constant
pressure. We tried inflating domes at different pressures. But in
experiments, as mentioned earlier, we noticed few to none of the domes
forming at lower pressure than \(50-100 Pa\). After optimizing for
different pressures we settled on using \(200Pa\) which allowed domes to
form and not delaminate.

We found that the areal strain of the dome would increase in first
\(3-5min\) of the pressure application. Later it would reach a plateau
in the strain till \(5-10 min\). Even though the dome to dome
variability is large in the strains ranging from 50\% to 300 \%. This
stabilization in strain indicates some kind of steady state achieved by
the tissue.

On closer examination of tension - strain curve of these domes, we see a
peculiar curve. The tension is non-monotonously changing with respect to
strain. It resembles a swish symbol of ``Nike''. The tension is
extremely high for low strains and then it is reduced to minimum around
areal strain of one, where dome is a perfect hemisphere. The dome
increases the tension later but the slope is relatively gentle as
compared to steep decline in tension for low strains.

This kind of material response would be strange for typical material
undergoing uniaxial stretching. However, in our case, the explanation of
the curve lies in the geometric constraint imposed by the dome system
and Laplace law. As the dome tension is dependent on radius of
curvature, as the dome inflates radius of curvature starts out very high
producing higher tension and it reduces to minimum with hemispherical
shape and then increases again. Radius of curvature, areal strain and
tension at inherently interconnected, such that at a constant pressure
we can write the expression of the curve.

\[ R = \frac{h^2 + a^2}{2h} \] \[ R = \frac{h^2/a^2 - 1}{2h/a^2} \]
\[ R = a\frac{\epsilon^2 - 1}{2\sqrt{\epsilon}} \]
\[ \sigma = \frac{Pa}{4} \frac{\epsilon^2 - 1}{\sqrt{\epsilon}}\] In
experiments, if we normalize the tension with the base radius we have
all the domes collapsing on one curve corresponding to a specific
pressure. We would call it `isobaric' curve.

In terms of physics, we understand that when the step pressure is
applied the dome suddenly inflates and has to undergo non-steady state
out of equilibrium stresses. It is very clear that this curve does not
represent the quasi static constitutive relation of the epithelia
tissue.

\hypertarget{constitutive-relation-of-epithelia}{%
\section{Constitutive relation of
epithelia}\label{constitutive-relation-of-epithelia}}

To obtain the real constitutive relation, ideally we should
quasi-statically increase strain or tension. In our system we can only
control the pressure. Increasing pressure slowly doesn't work with domes
as they don't delaminate at low pressures and if they do they are in
non-steady state regime where they inflate rapidly to higher strains.
The rational choice for us was to capture steady state corresponding to
different pressures.

We applied \(200 Pa\) pressure for \(5min\) till the dome achieves
steady state. Then we reduce the pressure by \(20Pa\) and wait for the
domes to reach steady state again. We continue repeating this step till
dome has been deflated totally. The tension strain curves this way
captured parts of isobarics as the dome passes through different
pressure steady states.

We obtained a constitutive relation which shows linear increase in the
tension with increase in strain for lower strains. However, for the
large strains the tension seems to be achieving a plateau consistent
with earlier MDCK dome studies. It is important to note that the dome to
dome variability is large. Also the tension recorded of \(4.5mN/m\) are
of the same order of magnitude as previous studies.

\hypertarget{dynamics-of-the-epithelia-domes}{%
\section{Dynamics of the epithelia
domes}\label{dynamics-of-the-epithelia-domes}}

To probe the dynamic material response of the domes, we performed cyclic
stretching experiments with the domes. We subjected domes to triangular
wave of pressure with magnitude of \(200Pa\) at three different
timescales. Considering the literature on cell remodeling and
experimental conditions, we chose 20s, 266s, and 2000s cycles.

Table1: Time period: 6s 60s 600s Rates: 0.2Pa/s 1.5Pa/s 20Pa/s

For fastest cycles, we observed domes progressively stretching more as
the cycles progresses till it reaches a steady state. We performed
experiment for 1200s (60 cycles) and noticed that the domes would
accumulate the strain progressively. It would stretch while loading and
unstretch while unloading but it would not go to zero strain after first
few cycles. In last cycles, we see that the dome oscillates between two
state of strains.

Similar response was seen in the moderate cycles, where domes are
stretched for 5 cycles each of 266s. The strain would accumulate in
first cycle itself with strains reaching higher than the fast case.
Also, after few cycles, it appears to have reached steady state.

Finally, for the slowest of all, 2000s cycles, we had difficult for
forming domes at lower pressure. As described earlier, the domes would
not detach till they reach critical pressure of 100-150Pa and then
inflate rapidly to really high strains of 200-300\%. However, it was
clearly visible that the strains were not accumulating. And there was no
difference in the maximum strains achieved in both cycles showing that
at this timescale the steady state is reached immediately.

One puzzling things we saw that in some cases of slow and moderately
fast cycles had delayed peak in strain when compared to maximum point of
pressure, indicating that when the pressure is reducing dome continues
to inflate and increase the strain.

\hypertarget{active-gel-tissue-model}{%
\section{Active gel tissue model}\label{active-gel-tissue-model}}

In parallel, to complement the understanding of the experiments, we
worked closely with Adam Ouzeri to develop an computational framework.
The general understanding of epithelial mechanics is that the actin
cortex viscoelasticity plays a critical role in sustaining deformations
at the timescale of seconds to minutes. Adam Ouzeri developed a
theoretical framework that bridges active gel models of actomyosin
cortex and 3D vertex models at tissue scales. In this model, each cell
is represented by a active gel surface accounting for physical aspects
of the cortex. We could assemble a tissue which is made with a
collection of these active gel surfaces.

In the model, the dynamics of the system is formulated through balance
of different potentials representing different active internal or
external forces and dissipation.

Briefly, the model incorporates the molecular dynamics of the actin
filament network along with myosin and crosslinker proteins through
following components:

\begin{enumerate}
\def\labelenumi{\arabic{enumi}.}
\item
  \textbf{Cortical thickness:} Here, the cell cortex is modeled as a
  hyperelastic membrane with cortical thickness (\(\rho_R\)). The
  deformation kinematics of this model is defined by mapping \(\varphi\)
  a cortical patch from a reference configuration \(\Gamma_0\) to
  deformed configuration \(\Gamma\) with metric tensor \(G_0\) to \(g\)
  respectively. As the material is dynamic, constituents of the network
  constantly changing, the reference configuration has to be dynamic.
  For this reason, there is a second reference configuration with its
  metric tensor, becomes a dynamic variable \(G\), similar to literature
  on nonlinear viscoelasticity. To summarize, the cortical thickness in
  reference configuration will change with the mapping change
  represented by Jacobian \(J_R\).
  \[ \rho_R(\xi, t) = \rho(x,t)J_R(\xi,t) \]
\item
  \textbf{Network elasticity:} This potential accounts for the free
  energy of the system undergoing deformation. The potential is
  dependent on the difference between in-plane strain (\(C\)) and the
  metric (\(G\)) written in format of hyperelastic potential (\(W\)). By
  using Neo-Hookean elastic potential, \(W\) will depend on two Lamé
  parameter, \(\lambda\) and \(\mu\)
  \[\mathcal{F} = \int_{\Gamma_R} \rho_R \ W(\mathbf{C,G})dS_R\]
\item
  \textbf{Dissipation:} The actomyosin network is known to remodel under
  tension, thus the released elastic energy can be accounted dissipation
  potential. It depends on a coefficient \(\eta\) equivalent to bulk
  viscosity, cortical thickness, and rate of metric tensor given by
  \(\dot{G}\)
  \[\mathcal{D} = \int_{\Gamma_R} \frac{\eta}{2}\ \rho_R \ \mathbf{\dot{G}}:\mathbf{\dot{G}} \ dS_R \]
\item
  \textbf{Active contractility:} Ultimately, this model is an active
  gel, the active part is included through an active power potential
  which adds energy in the system. It is dependent on the cortical
  tension and the rate of deformation tensor. The cortical tension is a
  active tension component of the network which is proportional to
  cortical thickness. \[ \gamma(\rho) =  \rho \xi\]
  \[ \mathcal{P} =  \int_{\Gamma} \gamma : \mathbf{d}  \ dS\]
\item \textbf{Turnover dynamics:} The turnover of the cortex is generalized
  into mass balance law, where network is constantly polymerizing and
  depolymerizing with cytosolic components. Here, it is assumed that
  there is a steady state cortical thickness.
  \[ \dot{\rho} + \rho \ tr(\mathbf{d}) = k_p\ C - k_d\ \rho \]
\end{enumerate}

The governing equations will be resulting from minimization of the
Rayleighian defined as following.

\[ \mathcal{R = \frac{dF}{dt} - D + P + P_e}\] Here, extra \(P_e\) is a
potential added because of the external forces and tractions. It was
assumed that the cell volume is conserved through the deformations.
Also, a mechanical barrier was implemented to limit strains beyond a
threshold. This was introduced by adding re-stiffening at large strains.
Physiologically, there are number of reasons by which cells would not
stretch perpetually, like activation of the intermediate filaments or
compression of the nucleus.

Broadly, we get three timescales associated with the model: Turnover
time (\(1/k_d\)), viscoelastic time (\(\eta / \lambda\)), and
viscoactive time (\(\eta / \xi\)). The model shows that the system
behaves like an active viscoelastic fluid, at shorter timescales it
behaves like a hyperelastic material and at longer timescales as
viscoelastic material.

Adam Ouzeri could implement this model in our system, where he would
create a digital twin of the monolayer with a cell monolayer. He could
also, have non adhesive regions which on application of pressure would
inflate in domes similar to our experiments.

\hypertarget{active-viscoelasticity-of-the-epithelia}{%
\section{Active viscoelasticity of the
epithelia}\label{active-viscoelasticity-of-the-epithelia}}

On repeating the simulations in conditions of the experiment, we could
understand the mechanics of the system. The digital domes on inflation
with constant pressure would inflate and achieve a steady state too.
Along with stretching of cell the cortical thickness decreases. However,
on holding constant pressure, after some time the strain is stabilized
when tissue tension is balanced by the forces we are applying in this
case is pressure

To evaluate the constitutive relation given by the model, digital dome
was inflated with different pressures to get isobaric curves and steady
state points. Alongside, a digital dome was inflated quasi-statically,
this is not feasible in the experiment. The goal was to see the
robustness of the model. We found that constitutive relation obtained
quasi-statically was consistent with the locus of the steady state point
in isobarics. The constitutive curve shows similar characteristics as
experiments with clear re-stiffening at large strains which we introduce
as a barrier mechanism.

We can interpret these results through the concept of the resting area,
It is an area of cell in a monolayer, where it is in steady state. When
perturbed from this state the actual area changes faster than the
resting area because of the viscoelastic behavior of the tissue. The
cell is able to dissipate the elastic stresses at the viscoelastic
timescales through remodeling, thus reaching steady state.

This effect of these timescales are more evident in the case of cyclic
stretching experiments. We can see clearly that the cells, when probed
faster than viscoelastic timescales, are accumulating strains because
they cannot dissipate the elastic stress. However, in case of slower
stretching the elastic stresses are able to be dissipated with
increasing area.

In digital dome, we can observe that in the last two cycles that the
resting area is almost overlapping the actual area for the slowest
condition. Because in this condition, we are changing pressure so slowly
at a rate of 0.2 Pa/s that cells can remodel and dissipate the elastic
stresses. The viscoelastic and turnover timescales are around 10-30s. So
over the period of 1000s dome stretch a lot.

On contrary, we see the other side where pressure is applied faster in
cycles of 20s, we see that the strains accumulate because there is not
enough time for cells to dissipate stored elastic energy. The
simulations show that the resting area marginally changes relative to
the actual area. It is worth noting that the simulations of creep
experiments, where tissue is stretch at constant tension, show strain
accumulation at visco-active timescale, where the contractility and
viscosity both act.

\newpage

\hypertarget{summary-and-discussion}{%
\section{Summary and discussion}\label{summary-and-discussion}}

In this work, we probed the mechanics of the epithelial tissue by
applying pressure at different rates. First, we found that at
application of step constant pressure of 200Pa domes dynamically inflate
and achieve a steady state in strain. Because of the spherical geometry
we found a non-monotonous tension-strain curve in response to constant
pressure. However, the true tension-strain curve revealed linearly
increasing tension with respect to strains for lower values, but at
higher strains the tension seems to be independent of the strains.
Second, we show that the domes when probed with fast changing pressure
increase and accumulate strain through the cycles. And reach the steady
state in later cycles. However, if stretched slowly the domes stretch to
high strains and do not accumulate strains.

We understood the epithelial tissue behavior through active viscoelastic
fluid model. Through experimental and computational framework, we could
probe the time dependent response of tissue to pressure. We could show
that the response of the domes to cyclic pressure is dependent on active
viscoelasticity.

The tissue stretches to balance the tissue tension with externally
applied pressure timescale and reaches steady state strain through
actively remodeling the cortex. The studies with digital domes show that
there are different timescales playing a role together to produce the
response of domes to pressure.

We could understand the model results in terms of multidimensional
maxwell model. Classical maxwell model is made up of a spring and a
dashpot representing elastic and viscous element. In this case, we can
imagine a similar model with two branches: one with spring and dashpot;
second with an active spring. If the stretching is done slowly the
elastic component (spring) will not stretch only the dash pot would
stretch. In other way if we stretch too fast the the viscous component
doesn't move only spring deforms. The interplay of cortical turnover,
crosslinkers, and the network reorganization allows for large
deformations and rapid shape changes.

In past, researcher have looked at the system in a similar fashion where
epithelial tissue was modelled by using viscoelastic models of springs
and dashpots. A particularly interesting model is done by Khalilgharibi
2019, which characterizes the response of suspended monolayer to stretch
and shows that the dynamics is similar to that of single cell because of
the role of actomyosin cortex. They use a model with two springs in
parallel, one of which can change its length dynamically. This explains
the relaxation of the monolayer, the active contractility of the cortex
changes the resting length of the active spring in the model.

Another study, found that the viscoelastic dissipation could explain the
shortening or elongation of the cell junctions in drosophila embryo.
They show that the dissipation occurs at the minute timescale at the
same timescale as myosin pulses. It is also interesting that they find
the actin turnover playing a key role in this dissipation.

There are very few in vitro techniques which can apply tensions and
strains on suspended cells. It is critical to understand that the
adherent monolayers behave differently than the suspended ones. The
matrix supporting the tissue plays a great mechanical role, in providing
extra stiffness and also altering the cytoskeletal structure of the
cells. However, in this thesis we focus only on actin cortex and the
short timescale (minutes).

The actin remodeling timescales are in order of tens of seconds. In our
system at this timescales we do not observe any cellular rearrangement
or division (very rarely). One of the reason that we didn't perform long
term experiments is because of the suspected involvement of the other
cytoskeletal components such as intermediate filaments.

In Latorre el al, they observed the activation of intermediate filaments
in extremely stretched cells (\textgreater300\%), which cause
re-stiffening and rescues cell from stretching too much. This motivated
the form of strain limiting mechanism imposed in the model. However, in
our experiments, we do not see any indication of the superelasticity all
the cells get super-stretched at the same time.

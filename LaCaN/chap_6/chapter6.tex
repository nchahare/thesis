\hypertarget{introduction}{%
\section{Introduction}\label{introduction1}}


To generate three-dimensional epithelial structures \textit{in vitro} from planar
epithelial monolayers, we chose to utilize an existing system of
epithelial domes developed by Ernest Latorre and improved by Ariadna
Marin. This system involves seeding a Madin-Darby canine kidney (MDCK)
cell monolayer on a substrate that is patterned with circular
non-adhesive regions. The cells invade these regions and form a cohesive
monolayer everywhere within 24 to 48 hours. Due to the active ion
pumping mechanism of the MDCK cells in the apical-to-basal direction,
the cells delaminate from impermeable substrates such as glass or soft
PDMS gel and form spherical cap structures in the circular patterns,
known as epithelial domes. Latorre and Marin demonstrated that they
could form a variety of structures with controlled shape and size,
ranging from circular to rectangular.

This system also enables the use of 3D traction force microscopy to
measure pressure. The technique involves measuring the deformation of a
soft PDMS gel embedded with beads to characterize the forces and
pressures applied by the cells on the substrate. This method offers an
innovative approach to measuring pressure compared to the previous
technique of puncturing epithelial domes with a microneedle. It allows
for the characterization of the rheology of epithelia and the discovery
of interesting material properties such as the superelasticity of cells
during stretching.

However, the formation of epithelial domes is dependent on the ion
pumping mechanism of the domes, making them spontaneous structures.
Therefore, the timescales for the dome stretching are not controlled.
This process can be marginally accelerated by a few hours through the
use of drugs like Forskolin, which can activate transepithelial channels
of NA+/K+/Cl-. However, to build and physically control the epithelial
structure, pressure control is necessary.

\hypertarget{monolayer-inflator}{%
\section{Monolayer inflator}\label{monolayer-inflator}}

Inspired by the concept of organ-on-chip microfluidic devices, we
considered them to be an ideal system for controlling pressure, cell
culture, and high-resolution imaging. For instance, the lungs-on-chip
device consists of two layers separated by a porous membrane, with one
channel in the top layer for epithelia and another for endothelia. The
device is assembled on a thin glass substrate, enabling high-quality
imaging.

Therefore, we conceived the idea of a monolayer inflator device, which
utilizes a two-layer microfluidic channel with one side for epithelial
monolayers and the other for the application of pressure. The epithelial
monolayer side is micropatterned with a protein that contains
non-adhesive or less-adhesive regions for dome formation. We
hypothesized that the cells would attach to the protein everywhere, even
in the less adhesive regions. When pressure is applied, the cells would
delaminate from the weakest point of adhesion and form a dome.

We chose to use PDMS material for building the microfluidic chip due to
its ease of use. We attempted to construct devices using plastic
stickers and photopolymerizable glue, but these attempts were
unsuccessful due to issues such as leakage and lack of biocompatibility.


\hypertarget{fabrication-of-the-device}{%
\section{Fabrication of the device}\label{fabrication-of-the-device}}

The structure of the device consists of four layers: glass, bottom
channel, porous membrane, and top channel. These layers are bonded
together using ozone plasma activation.

To enable high-quality imaging, the device must be mounted on a thin
glass slide. Although thicker glass slides are available, a glass slide
with a thickness of 1.5 is necessary for measuring the curvature of
domes or monitoring cell stretching.

The bottom channel must also be thin enough to ensure it is as close as
possible to the working distance of most confocal microscope objectives,
typically between \(200\mu m\) and \(1000 \mu m\). To achieve this, we
fabricated the bottom layer using a thickness of \(100 \mu m\). This
thickness is thick enough to handle manually but not too thin to cause
microfluidic problems with pressure and flow. We used a spin coating
method to fabricate a thin PDMS layer and then cut the channel out of it
using a desktop cutting machine.

The primary purpose of the porous membrane is to allow for pressure
application while preventing cells from passing through from the cell
channel to the pressure channel. We initially started with a \(10\mu m\)
membrane based on the literature. We attempted to use a \(100 \mu m\)
thin layer of PDMS with \(10\mu m\) pores using photolithography, but we
were unsuccessful due to producing \(10\mu m\) pillars with
\(100 \mu m\) height, which resulted in an aspect ratio that was too
high for us to get upright pillars. Therefore, we decided to use plastic
(PET) membranes with \(10\mu m\) pores. The thin (\(10\mu m\) thick)
plastic sheets were easy to handle, but we experienced bonding failures
and leakages due to membrane wrinkling.

Later, we decided to attach the PDMS thin layer to a small piece of
membrane. The middle PDMS thin layer was constructed with a \(1.2 mm\)
hole to expose the membrane to pressure, as this dimension is
approximately the size of the field of view of a 10X objective.

The top channel was designed as a large block with a \(5mm\) thickness
and a \(1mm\) thickness engraved channel. The thickness of the top
channel was chosen arbitrarily based on the need for the block to be
thick enough to plug in tubing for the application of pressure. We used
a 3D printer to create a mold with a channel, and four inlets were added
to the big block to accommodate two inlets for the bottom channel and
two inlets to seed cells.

Finally, all the layers were bonded together in two steps using an ozone
plasma cleaner. First, we bonded the glass to the bottom channel and
simultaneously bonded the middle layer to the top channel. Once these
layers were bonded, we then bonded them to each other with the membrane
sandwiched in the middle.


\hypertarget{protein-patterning-and-inverted-cell-culture}{%
\section{Protein patterning and inverted cell
culture}\label{protein-patterning-and-inverted-cell-culture}}

After a few trials, we realized that the entire device needed to be
contained in a Petri dish due to the cell culture medium. Placing the
glass slide in a larger Petri dish resulted in liquid flowing underneath
the glass, and any leakage during pressure application caused spillage
in the microscope. Consequently, we designed the setup to fit within a
glass-bottomed dish.

In the case of spontaneous domes, we seeded the domes on a soft PDMS gel
within a glass-bottomed dish. This allowed the top surface to be exposed
to any chemical treatment, including micro-contact printing, where a
PDMS block with topography was used to pattern adhesion proteins. To
achieve more control and flexibility in patterning proteins, we used the
PRIMO technique, which involved using an inverted microscope to etch
protein patterns into the substrate.

Since our setup was sealed and bonded, we could not use micro-contact
printing. Instead, we opted to use the PRIMO technique, which had been
optimized for glass and soft PDMS substrates. We had to optimize the
technique for our setup with a plastic substrate, which involved
increasing laser power and protein concentration. In the end, we
successfully patterned the proteins.

We also had to optimize the cell seeding. Unlike in other setups where
cells could be seeded in a dish, the channel required a higher
concentration of cells than typical spontaneous dome experiments. We
seeded \(30\times 10^6\) cells/ml for one hour and then washed away the
cells that did not attach within that hour

In our early experiments, we observed that while cells attached to the
porous membrane and protein patterns, there were very few dome
formations upon application of pressure. Additionally, the quality of
imaging was poor due to imaging through the porous membrane. In the top
channel, the cells were further away from the microscope objective, and
we noticed that cells were filtering through the membrane from top to
bottom, resulting in better protein coating on the bottom channel side
than the top.

To prevent cells from crossing the membrane, we decided to use a
membrane with smaller pore size, around \(400nm\). However, imaging the
green channel (\(488nm\)) through these pores was impossible. Therefore,
we decided to change the side of seeding cells from top to bottom, which
improved two things. First, imaging was better as the cells and dome
were closer to the objective. Second, cells were seeded on the good side
of the protein pattern.

We termed this ``upside-down'' cell culture, where we would flip the
device immediately upon seeding cells in the bottom channel to ensure
attachment on the membrane, not the glass. We had to thoroughly wash the
channel to prevent cells from attaching to the glass, which would
obstruct the imaging of the domes.

Despite these improvements, the ultimate challenge was ensuring that the
cell monolayer covered the non-adhesive regions. To resolve this issue,
we increased the protein concentration in these regions so that cells
could attach there weakly and detach first to form a dome.

\hypertarget{pressure-control}{%
\section{Pressure control}\label{pressure-control}}

After optimizing protein patterning, cell culture conditions, and
confocal microscopy, we turned our attention to pressure control.
Previous studies indicated that the pressure required to form a dome is
around \(100Pa\), which is equivalent to \(1cm\) of water column. Our
idea was to use hydrostatic pressure, as it is a low-pressure system.

In early trials, we used pipette tips to apply pressure. However, we
found that they were prone to bubbles and leaks. Therefore, we switched
to using Polytetrafluoroethylene tubing, which was connected to a
\(15ml\) tube. This tubing allowed us to match the height of the device
with the air-liquid interface in the tube, resulting in zero pressure on
the monolayer. By increasing the height of the tube by \(2cm\), we could
apply \(200Pa\) pressure to the monolayer, causing it to delaminate and
form domes.

However, we had to be careful of cavitation or bubbles getting trapped
in the cell channel. To prevent this, before the experiment we had to
place the media in a vacuum chamber to get rid of nascent bubbles that
could grow over time. Additionally, during tubing insertion, we could
introduce bubbles again. To address this issue, we used the two inlets
for each channel to flush the fresh media from the reservoir, ensuring
that there were no bubbles present.

To control the pressure, we used an automatic translation stage that
could be programmed to lift the reservoir. We measured the pressure by
tracking the height of the stage and the zero-pressure position. With
this stage, we could apply pressure in the range of
\(0\rightarrow 1500Pa\), and we could even apply negative pressure by
setting it lower. For our experiments, we used the range of
\(-200\rightarrow 1300Pa\).


\hypertarget{imaging-the-epithelial-domes}{%
\section{Imaging the epithelial
domes}\label{imaging-the-epithelial-domes}}

After optimizing the protein patterning, cell culture conditions, and
confocal microscopy, we were able to form domes according to the pattern
and control pressure at will. However, domes would not inflate right
away as they have to detach from the porous membrane. They would
delaminate at pressures between \(50-100Pa\), and then we could image
the dome using confocal microscopy. To image the dome, we used a 40x
objective with the membrane marked in the \(488nm\) channel and the
protein pattern in the \(644nm\) channel. The labeled adhesion protein
made it easier for us to track and anticipate the formation of the
domes.

Initially, we were mainly interested in spherical domes at constant
pressure to characterize the epithelial mechanics. We could monitor
pressure, cell shape, and tissue curvature easily, enabling us to
utilize Laplace's law to calculate tension. However, previous studies
have shown that the dynamics of the domes were slow and could be
recorded at a much slower rate of \(5\) minutes. In contrast, our domes
could be inflated and deflated in a matter of seconds, allowing us to
monitor them only by looking at the base of the dome, where the
monolayer would come in and out of view.

To study rheology, we need to be able to monitor the dynamic response of
the domes at faster pressure rates or shorter timescales. In this case,
we would only be interested in the dynamics of the dome strain and
curvature. Due to the symmetry of the dome, we could image only the
mid-section of the dome and obtain all the information needed to
characterize the material response of the dome. However, imaging the
dome stack with a step size of \(0.5\mu m\) and height of \(100 \mu m\)
took \(5\) minutes, which was much slower than we could deform the dome
by changing the pressure. Therefore, we needed to develop a new imaging
technique to monitor the faster dynamics of the domes.

With the line scanning mode of a Zeiss Airy Scan Microscope, we were
able to quickly image a single line of pixels across the midsection of
the dome and take a confocal z-stack along the height of the dome. This
provided us with a cross-section of the dome in a fraction of the time
of a normal stack. By enabling piezo stage movement, we were able to
image a \(100\mu m\) tall dome in just \(4s\), and could even track the
dome height evolution through a kymograph of the central part of the
dome. However, it is important to note that this form of imaging is
primarily useful for tracking dome strain and curvature, and the quality
of the cell images is often low. In cases where the fluorescent
expression of cells on the top of the domes was inadequate, the data was
much noisier.

\hypertarget{light-sheet-moli}{%
\section{Light-sheet MOLI}\label{light-sheet-moli}}

Later in my PhD, for observing cellular or sub-cellular changes, we
could utilize the light sheet microscope. The microscope in our lab, has
two immersion-upright objectives at 45 degrees to the horizontal plane.
With our expertise at fabricating devices, we were able to design new
setup which would have cells and porous membrane exposed to the top for
imaging.

We decided to invert the normal MOLI device and simplify the setup for
fabrication. For this, we needed a pressure channel and middle layer
with hole and porous membrane. The device had to be thick enough to plug
in the tubing. The whole device and channels are large so we could
fabricate the mold for this using normal 3D printer. We created a
ridge-like protrusion so that pressure channel and hole of seeding cells
is manufactured in one go. The bonding of the device was done by gluing
the device to a microscope slide with unpolymerized PDMS. We were able
to primo pattern the device by flipping it upside down. Seeding cells in
this setup is easier as the cell seeding part is exposed.

As expected, we were able to apply pressure with the same system as
before and form domes. We could see the features which were impossible
to see in other imaging strategies.Later in my PhD, we used a light sheet microscope to observe cellular or
sub-cellular changes. The microscope in our lab had two
immersion-upright objectives at 45 degrees to the horizontal plane. With
our expertise in fabricating devices, we designed a new setup that
exposed cells and porous membranes to the top for imaging.

To simplify the setup for fabrication, we inverted the normal MOLI
device. This required a pressure channel and middle layer with a hole
and porous membrane. The device had to be thick enough to plug in the
tubing. Since the device and channels were large, we could fabricate the
mold using a normal 3D printer. We created a ridge-like protrusion so
that the pressure channel and cell seeding hole could be manufactured in
one go. We bonded the device to a microscope slide using unpolymerized
PDMS. We were able to perform PRIMO patterning of the device by flipping
it upside down. Seeding cells in this setup was easier as the cell
seeding part was exposed.

As expected, we were able to apply pressure with the same system as
before and form domes. Using this imaging strategy, we were able to see
fast moving features that were impossible to see with other imaging
methods.


\newpage
\hypertarget{summary-and-discussion}{%
\section{Summary and Discussion}\label{summary-and-discussion}}

We have developed a microfluidic chip to generate 3D curved epithelia, utilizing a multilevel device consisting of two layers separated by a porous membrane. Seeding cells on the membrane in the bottom channel allowed for dome formation closer to the microscope objective, enabling high-quality confocal imaging. Hydrostatic pressure was used to control pressure under the dome dynamically, allowing for monitoring of cells and tissue behavior. Additionally, we developed imaging strategies to capture dynamics of these 3D structures faster using line scanning mode of confocal microscope or light sheet microscope.  

However, it is important to note that while the method of forming 3D epithelia described here may seem straightforward, it required many iterations of the device and other attempted methods that ultimately failed.  

By applying Laplace's law for spherical cap domes, we were able to measure pressure, tension, and curvature. As previous studies have shown, the epithelial tissue must adopt a spherical cap shape for circular footprint to maintain mechanical equilibrium. The uniform curvature and pressure imply uniform tension, requiring no knowledge of the tissue's material properties. However, in the case of non-spherical geometry, anisotropic stresses would be present, necessitating a computational model to solve an inverse problem and determine forces from geometry.  

The geometry of the domes is primarily controlled by the protein pattern, but delamination can still occur. In spontaneous domes, circular footprints were found to be the most common, while domes formed around sharp corners can blunt themselves through delamination. This must be taken into consideration when creating specific geometries.  

Tissue tension and adhesion forces also interact with each other. Cell-cell junctions are stronger than cell-substrate adhesion, so if tension at the base of the dome exceeds the adhesion forces, it can lead to detachment and delamination.  

Moreover, unintentionally, we have created a peeling system that allows us to observe tissue being peeled off from the substrate. If the dome remains spherical, we can calculate the forces required to break cell-substrate adhesion and identify the role of molecular components of focal adhesion. However, our primary interest lies in understanding the mechanics of epithelial tissue under controlled pressure.

\begin{Abstractcat}
	
	Les làmines epitelials formen estructures 3D especialitzades adequades als seus rols fisiològics, com ara alvèols ramificats als pulmons, tubs al ronyó i vellositats a l'intestí. Per generar i mantenir aquestes estructures, els epitelis han de patir deformacions 3D complexes en longitud i temps. La manera en què la forma epitelial sorgeix de tensions actives, viscoelasticitat i pressió luminal encara és poc entesa. Per abordar aquesta qüestió, hem desenvolupat un xip microfluídic i un marc computacional per dissenyar teixits epitelials 3D amb forma i pressió controlades. En aquest sistema, una monocapa epitelial es cultiva sobre una superfície porosa amb zones circulars de baixa adhesió. En aplicar pressió hidrostàtica, la monocapa es delamina i forma una estructura esfèrica a la zona circular. Aquesta forma simple ens permet calcular la tensió epitelial utilitzant la llei de Laplace. A través d'aquest enfocament, sotmetem la monocapa a una gamma de pressions luminals a diferents velocitats i, per tant, sondegem la relació entre deformació i tensió en diferents règims mentre seguim computacionalment la dinàmica de l'actina i el seu efecte mecànic a escala de teixit. Canvis de pressió lents en relació amb la dinàmica de l'actina permeten que el teixit acomodi grans variacions de deformació. No obstant això, sota reduccions sobtades de pressió, el teixit desenvolupa patrons de vinclament i plecs amb diferents graus de trencament de simetria per emmagatzemar l'àrea de teixit sobrant. Aquestes idees ens permeten modelar plecs epitelials a través d'uns plegaments dirigits racionalment. El nostre estudi estableix una nova estratègia per dissenyar esdeveniments morfogenètics epitelials.
	
\end{Abstractcat}

\cleardoublepage
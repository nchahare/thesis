\hypertarget{introduction}{%
\section{Introduction}\label{introduction5}}

In biology, epithelial structures come in all the shape and sizes. We
could envision all these structures as curved or folded epithelia.
Labeling epithelial structures is complicated, which really means folded
together. But at the same time, in developmental biology, development,
literally means unfolding of organism. In our system, we can generate 3D
dimensional epithelia with its inflation. In spirit of complicated
development processes, we decide to try folding and unfolding of the
domes in our experimental setup. Through understanding the mechanics of
the epithelial tissue, we realized that if the dome is depressurized at
extreme rates we could deflate the dome into folds. In this chapter, we
will discuss the mechanics of epithelial buckling and transforming domes
into folds.

\hypertarget{rapid-deflation-produces-a-buckling-instability}{%
\section{Rapid deflation produces a buckling
instability}\label{rapid-deflation-produces-a-buckling-instability}}

In the experiments with constant pressure, we found that the dome
achieves a steady state through remodeling the cytoskeleton. The
original area of the footprint increases to areal strains of more than
100\%, which more than twice of the original area. In cyclic stretching,
we found that the fast inflation-deflation would accumulate strains
through the cycles. Because of our total control over pressures, we
decided to subject the tissue to extreme rates of pressure, leaving no
time for tissue to relax and cause it to buckle.

The results from computational model, where we could track the digital
dome mechanical effect on tissue and cell scale, show that cortical
remodeling driven active viscoelastic dissipation allows dome to achieve
high strains. The cyclic stretching at fast rates show that the resting
area lags behind actual area. They suggest upon fast deflation this
discrepancy between actual and resting area could induce negative
viscoelastic stress leading to buckling. We observed that the digital
domes would inflate and reach the steady state then upon deflating
slower than remodeling scale would come down back into a flat monolayer.
In contrast, deflation faster than the remodeling timescales we get dome
buckling into folds.

In simulations, we found that it is essential to apply negative pressure
for undergoing buckling. This only affects the domes deflating fast as
the the pressure becomes negative while curvature remaining positive,
creating negative stresses. Whereas, the slowly deflating domes reaches
zero strain as pressure approaches zero.

To test this experimentally, we decided to systematically probe two
factors: first timescale for cytoskeletal remodeling and second
deflation rates. We designed the experiments with pressure profile
divided in three parts: 1. Linear increase in the pressure from 0 to 200
Pa in 10s 2. Holding constant pressure for different ``hold times'' 3.
Reducing pressure to -50Pa for different ``deflation rates'' We chose
the hold times and deflation rates considering the timescales associated
with actomyosin cytoskeleton.

Table2: Hold times: 6s 60s 600s Deflation rates: 0.2Pa/s 2Pa/s 20Pa/s
200Pa/s

It is worth noting that the buckling events were characterized
qualitatively. The assumption we made was that if the curvature of the
monolayer remains smooth and without kinks, we would consider it to be
not buckling. In some extreme cases where it was difficult for me to say
without biasing the data. I asked my colleagues, Thomas Wilson and Tom
Golde, to categories the data set. I performed the experiments for all
the conditions and quantified the data by tracking fraction of domes
buckling.

We witnessed buckling for all the conditions of hold times ranging from
6s to 600s for fastest deflation rate of 200 Pa/s. This confirmed our
hypothesis that the tissue undergoing rapid deflation would have to
reduce the area and compressive stresses will be introduced causing
buckling. On another hand, deflating slowly at rate of 0.2 Pa/s buckling
was observed rarely for all the hold times. This shows that the tissue
can remodel the cytoskeleton and change the area drastically to not
buckle.

As seen before, the longer ``hold times'' would give cells time to
remodel the cytoskeleton and the shorter ones won't. We learned that the
longer hold time resulted in higher strain and more likelihood of
buckling even for slower rates. One can see in the simple diagram that
the transition as the proportion of buckling domes increases towards the
faster deflation rates with longer hold time and decreases with slower
deflation rates with a shorter hold time.

We find that the domes for 6s hold time would have smaller strains as
compared to 600s hold time. This is consistent with the results from
domes subjected to the constant pressure. The strains increase takes
time as the dome remodels to balance the active tension with externally
applied pressure. However, it is interesting that we observe domes
buckling at 6s. In this condition, we wanted to inflate and deflate the
dome before it can remodel. By considering imaging speeds, we chose 6s
hold time, but it is still too slow for the remodeling timescales.

It is worth noting that there was lot of diversity in the buckling
patterns seen through a midsection of tissue imaged by line scan method.
For some domes the kinks in the folds were minor and in others they were
drastic, reminiscent of the buckling modes of the plates.

\hypertarget{multiscale-buckling}{%
\section{Multiscale buckling}\label{multiscale-buckling}}

The tissue buckling has been obvious in the confocal images. To get the
clear look at the shape of the cells, we created a variation of MOLI
adapted for light sheet microscopy. In higher resolution 3D images of
dome, we can see variety of thicknesses in the domes: thick part are
made by nucleus bulging out and very thin parts at cell periphery.

We repeated the same experiments as described previous section, but with
rapid deflation and long hold time. On buckling we discovered more
features than the tissue scale buckling. We noticed kinks and crimps at
the cellular and subcellular scales. On closer inspection, we classified
three levels of buckling:

Tissue scale buckling is clearly visible with cells collectively going
from uniform curvature to distorted shape. At this level, we see cells
deforming at larger scale and at the junctions.

At shorter scales, we noticed individual cells undergoing buckling. The
buckling cell double ups around itself with apical side on the top. The
length scale at which cell buckles is much shorter than the tissue. It
appears as the cell buckles as unit.

In some cases we see buckling at even shorter length scales, we call it
subcellular buckling, as the folds in membrane are distinct from the
cell level buckling. These are occurring in the thinnest parts of the
stretched cells where membrane buckles at much shorter wavelength.
Interestingly, these folds occur on the both, apical and basal, sides of
the cells.

Typically, in epithelial cells the membrane is tightly attached to the
cortex through membrane-cortex attachment proteins like Ezrin, Radixin,
Moesin. It is reasonable to believe that the subcellular buckling is due
to actin cortex buckling. To confirm this, we decided to image actin
cortex while buckling with SPY actin staining. We found that the actin
would follow exact shape of the membrane while undergoing buckling.

Even more interesting results for us were to see domes which not
buckling at all at tissue scale but would appear to be buckling at cell
or subcellular levels. Also these categories are not so strictly
separated, as there were multiple observation of tissue, cell, and
subcellular level buckling.

\hypertarget{generating-epithelial-folds}{%
\section{Generating epithelial
folds}\label{generating-epithelial-folds}}

After optimizing the buckling conditions, we decided to explore the
epithelial folds. While buckling the tissue area is squeezed into a
small area. This would produce folds and wrinkles in the monolayer. When
we monitored the base of dome deflating, we saw the tissue making
contact with the substrate certain regions first then others. This lead
to formation of folds in the places it made the contact last.

We looked at spherical domes of different sizes to see if there is any
pattern to these folds. We found that there were broadly three kinds of
folding pattern emerging.

\begin{enumerate}
\def\labelenumi{\arabic{enumi}.}

\item
  Accumulation along the periphery
\item
  Folds in the middle
\item
  Mixture of both
\end{enumerate}

First, for the domes with the footprint smaller than \(10000 \mu m^2\),
we observed repeatedly that most of the buckling would create an
accumulation around the periphery. When looking at the timelapse, it
gave us the impression of a donut-like structure. However, in
three-dimensional imaging of the folds, we could see that it was a
crescent shaped fold like a croissant. Fold was taller on one side than
other. For larger domes with footprint greater than \(30000\mu m^2\),
there are more instances of domes forming the network of folds in the
middle. We could get multiple folds connecting each other by forming
junctions. Finally, for the domes of intermediate size, we have mixture
of accumulation and folds. Although, the proportion of the folds along
the periphery decreases.

We find the same patterns as experiment when performing the same
deflation experiments with digital domes. The larger domes would produce
more radial folds and small domes form accumulation of on the side. In
the intermediate sizes we have mixture of both.

\hypertarget{forming-predictable-folds}{%
\section{Forming predictable folds}\label{forming-predictable-folds}}

We wondered about how the geometry of the dome would affect the pattern
formation. As we see with different sizes the buckle pattern changes.
The issue with spherical domes buckling is that the patterns, even
though beautiful, cannot be predicted because of the axis-symmetric
shape. We could possibly guide the folding by breaking the symmetry of
the domes.

Using MOLI, we generated ellipsoidal domes of different sizes. We found
that these domes would buckle into a fold along the major axis. Smaller
area domes would produce similar peripheral accumulation as spherical
domes. The larger domes produced a fold in the center along the major
axis. This reasserted that only larger domes would produce folds
regardless of the shape.

Finally, to create a controlled and more complex fold than a straight
line in the middle we decided buckle a dome with a triangular. We
expected that the vertices of the triangles could push the buckling
along the medians of the triangle. And as expected domes did buckle into
forming a Y-shaped network of the fold. When we repeated the triangular
and ellipsoidal shapes with digital domes, we found similar patterns.

On imaging the stability of these folded structures, we found that these
folds were stable for hours. We could image them more than 12 hours.
However, not all folds were alike, some folds would dissipate into the
monolayer. Whereas, others would last longer by forming attachments with
each other.

This could be a novel of producing folds. We envision tissue engineering
possibilities of the MOLI system by just controlling a few mechanical
parameters such as geometry and pressure.

\newpage
\hypertarget{summary-and-discussion}{%
\section{Summary and discussion}\label{summary-and-discussion}}

We used the device to create domes and then transform them into folds.
We characterized the buckling response with respect to actin remodeling
timescales. The domes buckle when deflated faster than they can remodel.
We found that the buckling occurs at multiple scales starting from
tissues then cells to actin cortex. Then we used this system to
understand the fold patterns with different shape and size domes. We
presented a novel approach of creating folds from pressurized structure.
Along with computational model, we unlocked the engineering potential
for the dome system to create controlled folds by just controlling
geometry and pressure.

As mentioned earlier, the mechanical instabilities are omnipresent in
biological systems. In particular, buckling phenomena has been observed
by researchers in MDCK epithelial monolayers through compressive
stresses applied by growth in confinement or direct application of
compression. Charras people show that the epithelial monolayer under
compression of more than 35\% strain buckles out of plane.
Interestingly, they show that the out of plane deformation is recovered
through active contractility within tens of seconds.

However, we are the first to be able to visualize the minute details of
the buckling process leading to insights in the behavior of tissue
architecture at multiple scales.

Mechanistically, we understand that these results are a consequence of
hierarchical nature of the epithelial tissue. There are various
components sustaining deformations and forces at different levels. Actin
cytoskeleton plays a key role in defining cell and tissue shape at
multiple scales. At the same time, cell monolayer can also be thought as
assembly of cells with their own surface tension and material
properties. Thus, it is expected that a material with different length
scales would also buckle at different length scales. If there is a local
weakness in a cell or subcellular feature, we could expect it to locally
buckle. For instance we only see subcellular level buckling in very thin
cells. This could also explain why we see no buckling at tissue scale
but see buckling at cellular or subcellular levels.

Moreover, it is fascinating to think about the short wavelength folds of
subcellular buckling. It is essential to point out that the actin
buckling is not new to the field. There are minimal models of actin
filaments with myosin motors on a lipid membrane, showing the myosin
induced contraction leads to actin filaments or shell buckling. In the
membrane-actin droplets, researchers show multiple modes in form of
buckling and wrinkling depending of the thickness. They found, counter
intuitively, thin shells undergoing buckling and thin shells producing
wrinkles in the membrane but not actin. It could relate to different
modes of buckling within a cell.

In context of tissue scale buckling, we can understand these results in
terms of modes of buckling in thin shells. The rate at which we are
deflating these tissues, our computational model suggest that the cortex
behaves like a hyperelastic material. If we repeat the same experiment
with elastic shells we would get similar results. In literature on thin
shell buckling, we find similar aspects where the folds and patterns
emerging when different sized shell are buckles.

Intuitively, we can imagine the slenderness, ratio of dome radius to
thickness, guiding the different modes of buckling. Thin shell with high
slenderness would lead to higher mode of buckling as compared to thicker
shell. However, our experiments go further than just understanding the
system but show that we program folds by minimally controlling two
parameters: geometry and pressure. In non spherical dome, Ariadna saw
for ellipsoidal domes the stresses were oriented in along the axes of
elliptical footprint. These anisotropic stress could orient the folding
in particular direction to generate a programmed fold. Like rectangular
footprint could create a mix of fold along the major axis as well as Y
junctions on the sides.
\hypertarget{introduction}{%
\section{Introduction}\label{introduction5}}

Epithelial structures in biology exhibit a wide range of shapes and
sizes, which can be understood as curved or folded epithelia. However,
accurately labeling epithelial folding is complicated, which really
means folded together. In the context of developmental biology, the term
``development'' refers to the unfolding of an organism, which highlights
the importance of understanding folding and unfolding processes.

In our experimental system, we can generate 3D epithelial structures
through inflation of the domes. By studying the mechanics of the
epithelial tissue, we have discovered that we can deflate the dome into
folds by rapidly depressurizing it. In this chapter, we will explore the
process of epithelial buckling and discuss how we have used this
knowledge to transform domes into folded structures.

\hypertarget{rapid-deflation-produces-a-buckling-instability}{%
	\section{Rapid deflation produces a buckling
		instability}\label{rapid-deflation-produces-a-buckling-instability}}

In our experiments involving constant pressure, we observed that the
dome reached a steady state through cytoskeletal remodeling.
Specifically, the original footprint area increased to areal strains
exceeding 100\%, more than double the original area. Moreover, when
subjected to cyclic stretching, we observed that rapid
inflation-deflation cycles accumulated strains in the tissue. Given our
precise control over pressure, we decided to expose the tissue to
extreme rates of pressure, allowing no time for relaxation and could
lead to buckling instability.

Our computational model enabled us to analyze the digital dome's
mechanical effects on both the tissue and cell scales, revealing that
active viscoelastic dissipation via cortical remodeling allowed the dome
to attain high strains. Cyclic stretching at rapid rates revealed a
discrepancy between actual and resting area, suggesting that fast
deflation could induce negative viscoelastic stress, leading to
buckling. We observed that the digital domes inflated and reached a
steady state, then deflated at a rate slower than the remodeling scale,
eventually returning to a flat monolayer. However, when deflated more
rapidly than the remodeling timescales, the domes buckled into folds.

In our simulations, we found that negative pressure was necessary for
inducing buckling, which only affected domes deflating rapidly.
Specifically, the pressure became negative while the curvature remained
positive, resulting in negative stresses. By contrast, slowly deflating
domes reached zero strain as the pressure approached zero.

To systematically test our hypothesis regarding the factors affecting
buckling events, we designed experiments with pressure profiles
consisting of three stages. Firstly, we initiated a linear increase in
pressure from 0 to 200 Pa over a period of 10 seconds. Secondly, we
applied constant pressure for varying ``hold times,'' which were chosen
based on the timescales associated with actomyosin cytoskeletal
remodeling. Finally, we decreased the pressure to -50Pa at varying
``deflation rates.''

It is important to note that we relied on qualitative characterization
of buckling events, assuming that smooth and continuous curvature of the
monolayer indicates the absence of buckling. However, in cases where it
was difficult to make an unbiased judgment, we enlisted the help of
Thomas Wilson and Tom Golde to categorize the data. We performed the
experiments for all conditions and quantified the data by tracking the
fraction of domes that underwent buckling.

Our results showed that buckling occurred for all hold times, ranging
from 6s to 600s, at the fastest deflation rate of 200 Pa/s. This
confirmed our hypothesis that rapid deflation would induce compressive
stresses and cause buckling. On the other hand, slow deflation at a rate
of 0.2 Pa/s rarely led to buckling, regardless of the hold time. This
suggests that the tissue can effectively remodel its cytoskeleton and
adapt to drastic changes in area to avoid buckling.

As seen in previous experiments, longer ``hold times'' allowed for more
cytoskeletal remodeling, resulting in higher strains and a higher
likelihood of buckling, even for slower deflation rates. Results plotted
in a phase diagram illustrates the trend: as hold time increases,
buckling becomes more likely at faster deflation rates and less likely
at slower deflation rates with a shorter hold time.

Interestingly, the data showed that domes with a 6s hold time had
smaller strains compared to those with a 600s hold time, which is
consistent with results from domes subjected to constant pressure.
Strain increase requires time for the dome to remodel and balance active
tension with the externally applied pressure. However, even at 6s hold
time, we still observed buckling. Expectation for this condition was to
allow us for inflation and deflation of the dome before it could
remodel. We selected a 6s hold time based on imaging speeds, but it
remained too slow for the remodeling timescales.

It is important to note that we observed a wide variety of buckling
patterns when examining midsections of the tissue through a line scan
method. Some domes exhibited minor kinks in the folds, while others
showed drastic buckling modes similar to those of plates.

\hypertarget{multiscale-buckling}{%
	\section{Multiscale buckling}\label{multiscale-buckling}}

The observation of tissue buckling was evident in the confocal images.
In order to gain a clearer understanding of the shape of the cells, we
adapted a variation of MOLI for use with light sheet microscopy. The
higher resolution 3D images of the dome revealed a variety of
thicknesses, with thicker regions caused by the bulging of the nucleus
and very thin regions at the cell periphery.

To further investigate the phenomenon of buckling, we repeated the
experiments described in the previous section, this time with rapid
deflation and a long hold time. We discovered additional features beyond
tissue-scale buckling, including kinks and crimps at cellular and
subcellular scales. Upon closer inspection, we classified three levels
of buckling:

At the tissue scale, buckling was visible with cells collectively
transitioning from a uniform curvature to a distorted shape. At this
level, we observed cell deformation at a larger scale, including at the
junctions between cells.

At shorter scales, we observed individual cells undergoing buckling. The
buckling of the cell resulted in the cell doubling up around itself,
with the apical side at the top. The length scale at which cell buckling
occurred was much shorter than that observed at the tissue level. It
appears as though the cell buckles as a single unit.

In some cases, buckling occurred at even shorter length scales, which we
refer to as subcellular buckling, as the folds in the membrane were
distinct from the cell level buckling. These folds occurred in the
thinnest parts of the stretched cells, where the membrane buckled at
much shorter wavelengths. Interestingly, these folds occurred on both
the apical and basal sides of the cells.

In epithelial cells, the membrane is typically attached to the cortex
through membrane-cortex attachment proteins such as ezrin, radixin, and
moesin. It is reasonable to assume that the subcellular buckling
observed in our experiments is due to actin cortex buckling. To confirm
this, we imaged the actin cortex while the dome was undergoing buckling
using SPY actin staining. Our results showed that the actin cortex
followed the exact shape of the membrane during buckling.

We also observed interesting results where some domes did not appear to
be buckling at the tissue scale, but were still exhibiting buckling at
the cell or subcellular level. These categories are not strictly
separated, as we observed multiple instances of tissue, cell, and
subcellular level buckling.

\hypertarget{generating-epithelial-folds}{%
	\section{Generating epithelial
		folds}\label{generating-epithelial-folds}}

After optimizing the buckling conditions, we conducted experiments to
investigate the formation of folds and wrinkles in epithelial cells. As
the tissue area is squeezed into a small area during buckling, the
tissue makes contact with the substrate in certain regions first,
leading to the formation of folds. We monitored the base of the dome
deflating and found three broad patterns of folding emerging:
accumulation along the periphery, folds in the middle, and a mixture of
both.

For smaller domes with a footprint less than \(10000 \mu m^2\), we
repeatedly observed that most of the buckling resulted in an
accumulation around the periphery, creating a crescent-shaped fold
resembling a croissant. When looking at the timelapse from the base of
the dome, it gave us the impression of a donut-like structure. In
contrast, larger domes with a footprint greater than \(30000\mu m^2\)
tended to form a network of folds in the middle, with multiple folds
connecting to each other and forming junctions. Intermediate-sized domes
exhibited a mixture of accumulation and folds, although the proportion
of folds along the periphery decreased.

Interestingly, we observed the same folding patterns when performing
deflation experiments with digital domes. Larger digital domes produced
more radial folds, while smaller digital domes formed accumulations on
the side. Intermediate-sized digital domes exhibited a mixture of both
patterns

After optimizing the buckling conditions, we embarked on exploring
epithelial folds. We have been imaging only the cross-section of the
dome to capture the fast dynamics. So, we can only see the buckling in
form of the squiggly lines. However, these buckling events are
three-dimensional. During tissue buckling, the squeezed area resulted in
the formation of folds and wrinkles in the monolayer. Monitoring the
base of the dome deflating, we observed that the tissue made contact
with the substrate in certain regions first and others later. This led
to the formation of folds in the regions where it made contact last.

To investigate if there was any pattern to these folds, we looked at
spherical domes of various sizes. Broadly, we observed three types of
folding patterns emerging:

\begin{enumerate}
	\def\labelenumi{\arabic{enumi}.}
	\item
	Accumulation along the periphery
	\item
	Folds in the middle
	\item
	A mixture of both
\end{enumerate}

For domes with a footprint smaller than \(10000 \mu m^2\), we repeatedly
observed that most of the buckling resulted in an accumulation around
the periphery. The confocal timelapse from the base gave the impression
of a donut-like structure, but three-dimensional imaging of the folds
revealed a crescent-shaped fold like a croissant, taller on one side
than the other. For larger domes with a footprint greater than
\(30000\mu m^2\), we observed more instances of domes forming a network
of folds in the middle, with multiple folds connecting each other by
forming junctions. Finally, for domes of intermediate size, we observed
a mixture of accumulation and folds, although the proportion of folds
along the periphery decreased.

Interestingly, we observed the same folding patterns in our digital
domes when performing the same deflation experiments. Larger digital
domes produced more radial folds, and small digital domes formed an
accumulation on the side. Intermediate-sized digital domes showed a
mixture of both patterns.

\hypertarget{forming-predictable-folds}{%
	\section{Forming predictable folds}\label{forming-predictable-folds}}

We were curious about how the geometry of the domes could affect the
pattern formation of folds. Although we observed that different sizes of
spherical domes produced different beautiful buckling patterns, the
axis-symmetric shape made it difficult to predict the patterns. To
address this, we decided to generate ellipsoidal domes of different
sizes using MOLI. Interestingly, we found that the ellipsoidal domes
would buckle into a fold along their major axis, with smaller
ellipsoidal domes producing a similar peripheral accumulation as
spherical domes and larger ellipsoidal domes producing a fold in the
center along the major axis. This suggested that only larger domes,
regardless of their shape, could produce folds.

To further explore the possibility of creating more complex folds, we
decided to buckle a dome with a triangular shape, anticipating that the
vertices of the triangle could push the buckling along the medians of
the triangle. As expected, the domes did buckle into forming a Y-shaped
network of the fold. We also repeated the triangular and ellipsoidal
shapes with digital domes and found similar patterns.

We also investigated the stability of these folded structures and found
that they were stable for hours and could be imaged for more than 12
hours. However, not all folds were alike, as some would dissipate into
the monolayer while others would last longer by forming attachments with
each other. Interestingly, if immediately inflated these folds would
unfurl themselves into a dome again.

These results suggest that the MOLI system could provide a novel way of
producing folds, with potential applications in tissue engineering, by
simply controlling a few mechanical parameters such as geometry and
pressure.

\hypertarget{summary-and-discussion-1}{%
	\section{Summary and Discussion}\label{summary-and-discussion-1}}

We utilized our device to generate dome from a flat monolayer then
transform it into folds, alongside investigating the buckling response
in relation to actin remodeling timescales. We discovered that buckling
occurs at various scales, starting from the tissue level down to the
actin cortex of individual cells, and is triggered when deflation occurs
more rapidly than actin can remodel. We then explored the patterns of
folding that emerge from different sized and shaped domes, and proposed
a new method of creating controlled folds from planar monolayer. With
the aid of computational models, we demonstrated the engineering
potential of the dome system to produce structured folds by manipulating
the geometry and pressure.

As mentioned earlier, mechanical instabilities are ubiquitous in
biological systems, and the phenomenon of buckling has been observed in
MDCK epithelial monolayers through various methods, such as growth in
confinement or direct application of compression. Charras et
al.~demonstrated that compressive stress greater than 35\% strain can
cause epithelial monolayers to buckle out of plane, and showed that
active contractility can recover the out-of-plane deformation within
tens of seconds.

Our study, on the other hand, is the first to offer visual insights into
the minute details of the buckling process and its implications for
tissue architecture at multiple scales.

From a mechanistic perspective, our result are a consequence of the
hierarchical structure of the epithelial tissue, which comprises various
components that sustain deformations and forces at different levels.
Notably, the actin cytoskeleton plays a critical role in defining the
shape of cells and tissues at multiple scales. Additionally, a cell
monolayer can be considered as an assembly of cells with their own
surface tension and material properties, indicating a material with
different length scales would buckle at different length scales.
Therefore, if there is a local weakness in a cell or subcellular
feature, it is expected to locally buckle. For instance, we only
observed subcellular-level buckling in very thin cells while overall
tissue doesn't buckle.

Furthermore, the short-wavelength folds resulting from subcellular
buckling are intriguing to consider. It is worth noting that actin
buckling is not a new phenomenon in the field, and there are minimal
models of actin filaments with myosin motors on a lipid membrane
demonstrating that myosin-induced contraction leads to actin filament
buckling. In membrane-actin droplets, researchers have reported multiple
modes in the form of buckling and wrinkling, depending on the thickness.
Interestingly, they found that thin shells undergo buckling and thin
shells produce wrinkles in the membrane but not in actin, which could
indicate different modes of buckling within a cell.

In the context of tissue-scale buckling, our results can be understood
in terms of modes of buckling in thin shells. Our computational model
suggests that the cortex behaves like a hyperelastic material, as
evidenced by the rate at which we are deflating these tissues. Similar
results would be obtained if we repeated the experiment with elastic
shells. The literature on thin shell buckling reveals similar aspects,
such as the folds and patterns that emerge when different sized shells
buckle.

The slenderness, defined as the ratio of dome radius to thickness,
intuitively guides the different modes of buckling. Thin shells with
high slenderness would lead to a higher mode of buckling compared to
thicker shells. However, our experiments go beyond understanding the
system and show that we can program folds by minimally controlling two
parameters: geometry and pressure. For non-spherical domes, anisotropic
stresses are observed along the axes of the elliptical footprint for
ellipsoidal domes. These stresses can orient the folding in a particular
direction to generate a programmed fold, such as Y junctions on the
sides of a rectangular footprint.

In summary, this thesis presents a novel experimental system that allows
us to inflate epithelial domes and deflate them into tubes. We
demonstrate that the timescales of actomyosin cytoskeleton remodeling
play a key role in this transformation. By controlling the geometry of
the epithelia and the rate of deflation, we show that we can engineer
epithelial folds of desired geometry.
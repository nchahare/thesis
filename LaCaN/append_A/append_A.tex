\section{Fabrication of microfluidic devices}
 
Polydimethylsiloxane (PDMS) gels (Sylgard PDMS kit, Dow Corning) were used to make the microfluidic devices. PDMS was synthesized by mixing the curing agent and elastomer in 1:9 weight ratio. This mixture was centrifuged for 2min at 900rpm to remove air bubbles. The unpolymerized PDMS was poured into a mold or spun to obtain the desired shape. 

There are four parts to the device (figgg). First is the top block, a thick PDMS block with four inlets and one channel for the application of hydraulic pressure. The second is a $200 \mu m$ thin PDMS layer with a $1.2mm$ diameter hole in the center with a $400nm$ porous membrane (Polycarbonate filtration membrane $0.4\mu m$, Whatman membranes) attached to it. The third is another $200 \mu m$ thin PDMS layer with a channel for seeding the cells. Lastly, all these PDMS parts are attached to, the fourth part, a glass-bottomed 35mm dish ($35mm, no. 1.5\#$ coverslip thickness, Cellvis). 

The top block was made using replica molding in a 3D printed mold. This mold was 3D printed with vat polymerization and a digital light processing 3D printer (Solus DLP 3D Printer with SolusProto resin). The mold’s surface was then silanized using Trichlorosilane (Trichloro(1H,1H,2H,2H-perfluorooctyl) silane, Merck) for preventing adhesion with unpolymerized PDMS. PDMS was poured into the mold and degassed for one hour. PDMS is cured with a hot plate at $100\circ C$ for $30min$. Once cured, PDMS is removed, cut into devices, and punched with $1.5mm$. $200\mu m$ thin PDMS layers were made by spin coating $4.5ml$ unpolymerized PDMS on a $15cm$ dish at $500rpm$ for $1min$. These dishes were incubated in an oven at $80\circ C$ to polymerize for 12hr. These thin sheets were cut into the parts of devices using a Silhouette cutting machine (Silhouette Cameo 4, Silhouette America). The sheets were attached to a Silhouette cutting mat and then Silhouette software was fed with the pattern of the device layers. A sharp cutting tool in the machine cut the PDMS along the pattern. These cut PDMS were peeled off with help of $70\%$ ethanol. 

These devices are assembled with the aid of ozone plasma cleaner (PCD-002-CE, Harrick Plasma). Glass bottomed dishes and thin PDMS layers with cell channels were treated for $1 min$ under plasma. Then bonded together by placing the layers in contact for $2 hr$ at $80 \circ C$. Similarly, the top block and thin membrane with porous membrane were also bonded. These layers were later bonded together again using plasma cleaner.

\section{Patterning protein on the device}
The devices were filled with $96\%$ ethanol for removing air bubbles. Then, devices are treated with $5\%$ v/v (3-aminopropyl) triethoxysilane (Merck) diluted in $96\%$ ethanol for $3min$ and rinse three times with $96\%$ ethanol. Later the devices were filled with MilliQ water to remove ethanol traces. PRIMO (Alveole Lab) was used to pattern adhesion-promoting protein. For this setup, devices were incubated with PLL (Poly-L-lysine solution, Merck) for $1hr$, subsequently with SVA PEG ($50mg/ml$ in $8.24pH$ HEPES) for $30min$, and rinsed with HEPES. Before using PRIMO, devices were filled with a photoinitiator. Desired protein pattern was loaded into the PRIMO software (Leonardo, Alveole Lab). 

PRIMO uses a microscope to shine the laser in the specific region according to the loaded pattern to cut PEG chains. After the PRIMO process, the samples were rinsed with phosphate-buffered saline (PBS, Merck). Then the samples were filled with fibronectin and fibrinogen ($100\mu g/ml$ Fibronectin in $2\%$ Far-red fibrinogen solution in 1X PBS) solution for $5 min$. Then samples were rinsed again with 1X PBS. Fibrinogen labels the fibronectin with Far-red signal to image the coated protein pattern and allows for tracking the position of the domes. The PRIMOed samples can be stored at $4\circ C$ for two to three days before seeding cells.

\section{Cell culture in the device}

To image cell shape and tissue structure Madin-Darby Canine Kidney (MDCK) cells expressing CIBN-GFP-CAAX were used for the experiments. CIBN-GFP-CAAX labels the plasma membrane. These cells were cultured in Dulbecco’s Modified Eagle Medium (DMEM, Gibco Thermofisher) with $10\%$ v/v fetal bovine serum (FBS, Gibco, Thermofisher), L-glutamine (Thermofisher), $100\mu g/ml$ streptomycin and penicillin. Cells were incubated at $37\circ C$ with a $5\%$ $CO_2$ condition. 

Before seeding cells in the device, it is filled with a cell culture medium. Cells are trypsinized and diluted at a concentration of $25-30\times10^6 cells/ml$. The cell channel of the device is filled with $30\mu l$ of cell solution and incubated for cell adhesion. After one hour of incubation, devices are rinsed with media to remove unattached cells. Devices were kept $24hr$ in the incubation for the growth of a monolayer before the experiment. It is important to note that the inverted epifluorescence microscope is need to see the cells. The bright-field microscope can not be used for visualizing cells because of the porous membrane.

\section{Staining actin with SPY-actin}

To observe the dynamics of the cortex, we used SPY555-actin (Spirochrome), a bright dye optimized for quick labeling of F-actin in live cells with low background. To prepare the 1000x solution, we added  $50\mu l$ of anhydrous DMSO to the stock SPY555-actin. We then added $1 \mu l$ of the 1000x solution to $999\mu l$ of cell culture medium. The resulting solution was introduced into the microfluidic chips and left in the incubator for 2 hours before imaging.

\section{Fabrication method for the Light-Sheet MOLI device }

The devices used with the light-sheet microscope consisted of a single PDMS block bonded to a glass microscope slide ($76 \times 26 mm$, RS Components BPB016). The blocks were made using a 3D printed mold (Ultimaker 3 with Ultimaker PLA Printer Filament 1616). PDMS was mixed, centrifuged, degassed, and cured as described above for the normal devices. Once cured, the PDMS was removed, cut into individual devices and punched with a 1.5mm biopsy punch. The PDMS blocks were then attached glass slides using a thin layer of unpolymerized PDMS, that was coated onto the glass slides using a spatula. The devices were then kept on a hotplate at $100\circ C$ for $30min$ to allow the PDMS bonding to fully cure. The $400nm$ porous membranes were then attached to the devices. The edges of the membrane were carefully dipped into unpolymerized PDMS, before being placed flat on the top of the device. Particular care was taken to ensure the center of the membrane over the punched pressure-application hole remained free of PDMS. The devices were then kept at $65\circ C$ for an hour to allow the PDMS bonding to fully cure.

\section{Device protein patterning and cell culture in Light-Sheet device}

The light-sheet devices were protein patterned and cell cultured using the same methods and steps as outlined above for the normal devices, with the one minor addition of the use of a simple PDMS and glass cap for a few critical steps. The porous membrane for pressure application, and thus the site of protein patterning and cell seeding, for the light-sheet devices is exposed and on the top side of the devices. This mostly allowed for easy application of reagents as a droplet could be applied and aspirated directly. 

However for the more sensitive steps in the procedure, a simple PDMS and glass device was used to create a temporary covered channel over the porous membrane to regulate the procedure and ensure the treatment of the devices was highly standardized. Specifically, the cap was used for the application of photoinitiator during PRIMO, and for the application of cell solution during cell attachment. The caps were fabricated using $2cm\times2cm$ squares of a $400\mu m$ thick PDMS layer, with a keyhole shape cut in from the side. Each PDMS piece was then stuck to a $18mm$ diameter coverslip ($18mm$, $1\#$ Cover glasses circular, Marienfeld 0111580) using the surface tension of the liquid. The experimental apparatus and measurements for the light-sheet devices were the same as the normal devices as outlined above.

\section{Application and measurement of the pressure}

The pressure is applied via hydrostatic forces similar to the previous studies \cite{choudhury2022, palmer2021}. The two channels in the chip were separated by the porous membrane. Cells are on the bottom side of the membrane. The pressure in the channel (top side of the membrane) is used to inflate the structures on the top. This channel has one inlet and one outlet for removing bubbles. The inlet is connected to a $35 ml$ reservoir of cell culture medium (in a $50 ml$ falcon tube) by tubing (PTFE Tubing $1/16  inch$ OD for Microfluidics, Darwin microfluidics) and the outlet is connected to a shutoff valve (Microfluidic Sample Injection / Shut-off Valve, Darwin microfluidics). Once bubbles are removed, closing the valve would apply the pressure on the basal side of the cells according to the difference between the height of the fluid level. All tubings are connected to the chip with a steel insert (Stainless steel 90deg Bent PDMS Couplers, Darwin microfluidics). We are able to find zero by matching the height of the device to the liquid and air interface in the reservoir. This is confirmed with the experiments, where on applying pressure domes form but on slow reduction in pressure to zero causes domes to deflate.

\section{Confocal Microscopy}

For timelapse imaging of domes at a larger time interval (> 1 min), an inverted Nikon microscope with a spinning disk confocal unit (CSU-W1, Yokogawa) was used with Nikon 40x, 20x, and 10x air lenses. For shorter time intervals (< 10 s), a Zeiss LSM880 inverted confocal microscope was used with laser scanning mode. Fast imaging was enabled by imaging a single line in the middle of the dome. 

\section{Light-sheet microscopy}

The imaging of the light-sheet devices was done with a dual-illumination inverted Selective Plane Illumination Microscope (diSPIM) (QuVi SPIM, Luxendo, Brucker) with Nikon 40x immersion lenses (Nikon CFI Apo 40x W 0.8 NA NIR water immersion objective). For the buckling experiments, only single objective illumination and detection was used for the fast imaging of $2 s/frame$. 

\section{Quantification of the dome areal strain and tension}

As mentioned earlier, the domes were imaged in 3D with confocal microscopy. We used ImageJ to manually section the dome in the middle in the YZ plane, XZ plane is a plane parallel to the monolayer, with Reslice function along the Z axis. This section was used to calculate the height (h), radius of curvature (R), and base radius (a). Strain ($\epsilon$) and tension ($\sigma$) were calculated as,
$$\epsilon = \frac{h^2}{a^2}, \ \ \ \ and \ \ \ \ \sigma = \frac{\Delta P R}{2}.$$
The raw data was extracted in ImageJ and then MATLAB was used to compute and plot the strain and tension.

\section{Analysis of the kymographs}

For cyclic pressure or buckling experiments, the domes were imaged at low resolution and high noise levels to capture fast dynamics. The previous method of manually quantifying each time point is not feasible. Thus, we used the ImageJ function of the Reslice function along the time axis. We resliced it along the Y-time axis in the middle of the dome, such that we get a kymograph of height as a function of time. Also, we performed the reslicing along the XT axis at the plane of the monolayer, such that we get the kymograph of the base radius with respect to time. These kymographs were in form of images save manually with ImageJ. 

A custom-built MATLAB code was used to digitize the kymographs, where maximum intensity along each time was considered as the current dome height position. The first $30 s$ of the experiment pressure is zero, so the unstretched monolayer position is determined from those time points. Dome height is calculated with the difference between the current position and the initial position. Base radius is calculated similarly by subtracting two sides. The radius of curvature is calculated using the relation between the base and height of the dome as
$$R=(h^2+a^2)/2h.$$


\section{Qualitative analysis of the buckling event}

Whether domes are buckling or not was determined manually checking every frame during the deflation. If dome maintains the smooth circular geometry in XZ plane during the deflation, we mark the dome as “not buckling”. However, if the dome has a visual discontinuity in the curvature or a kink it is then considered to be “buckling”.

Imaging the fast events in XY plane was done in an ad hoc manner. To capture the folds, the dome as imaged closer to the apical surface of the monolayer. The type of fold was determined by carefully observing the way which monolayer makes contact with the imaging plane. If there is one point of contact in the center and spreads outwards, it is considered as accumulation along the periphery. In case where there are multiple points of contact and they all join in the middle, it is considered as a network of folds.
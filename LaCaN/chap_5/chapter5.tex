\hypertarget{what-is-to-be-done}{%
\section{What is to be done?}\label{what-is-to-be-done}}

Morphogenesis is a process of deformation or growth of the tissue under the combination of endogenous and exogenous mechanical forces that include contractility of the epithelium itself and the surrounding matrix as well as hydraulic pressure from the lumen. These stresses are applied to different material components of the tissues, such as cells and the extracellular matrix, that display distinct viscoelastic properties and remodeling time scales. Understanding how the complex interplay between tissue stresses and viscoelastic properties gives rise to specific morphogenetic events in vivo poses outstanding technical and conceptual challenges. These include difficulties to disentangle the relative role of the distinct components involved in a system, the lack of tools for quantitative measurements of stresses and mechanical properties, and the inability to impose controlled stresses over a broad range of amplitudes and rates.~

As a complementary strategy, bottom-up approaches aim at understanding the role of each component of the system and its morphogenetic potential, with the ultimate goal of building complexity through rational engineering of the building blocks that form functional tissue. These approaches have been successful at engineering elementary morphogenetic processes such as epithelial bending or buckling. However, despite the emerging success of bottom-up approaches, we still lack tools to simultaneously measure and control the shape and stress of 3D epithelia. In addition, we lack computational models that integrate cellular and tissue shape with the subcellular determinants of epithelial mechanics such as the contractility, turnover, and viscoelasticity of the actomyosin cortex.

The focus of this thesis is to investigate the mechanics of epithelial tissues. Understanding the principles that govern tissue form and function is critical for two main reasons. Firstly, it allows us to comprehend the fundamental physical rules of biology. Secondly, it provides inspiration for new engineering tools and design principles. We use cutting-edge technologies such as 3D printing, microfluidics, and 3D
cell cultures to individually control morphogenetic driving factors. This approach enables us to study tissues from a material science perspective, which is particularly useful for probing the intricate mechanisms involved in the generation of forces and shape changes at the cellular and tissue levels. Furthermore, this approach has the potential to lead to the discovery of emergent phenomena and enable the building of novel tissue forms and assemblies.

\hypertarget{objectives}{%
\section{Objectives}\label{objectives}}

\subsection*{General aim of the thesis} 

Explore the mechanics of epithelial layers subjected to controlled pressure.

\subsection*{Specific aims of the thesis} 
General aims are divided into specific goals:

\begin{enumerate}
\def\labelenumi{\arabic{enumi}.}
\item  Develop a new technology to build a three-dimensional epithelia by using controlled hydrostatic pressure.
\item  Characterize material response of the pressurized epithelial tissue.
\item  Explore the mechanics of epithelial folds.
\end{enumerate}

\hypertarget{thesis-outline}{%
\section{Thesis outline}\label{thesis-outline}}

Results are presented in Part 2 with four chapters addressing all the aims and conclusions.

\begin{itemize}
\item Chapter 6 will detail the construction of an experimental system for physically controlling epithelial monolayers. This chapter will showcase the main result of the PhD, a novel microfluidic system that generates 3D epithelia with controlled pressure and shape. I will explain the motivation for the device and provide a summary of failed or attempted methods used in constructing the device.
\item Chapter 7 will focus on using the microfluidic device to understand epithelial mechanics. In this chapter, I will report the results of rheological experiments and relate them to the computational framework. We show that the shape and rheology of the epithelia are driven by the viscoelasticity of the actomyosin cortex. Additionally, we will make predictions related to tissue buckling.
\item Chapter 8 will describe the buckling instability in pressurized epithelia. We found that rapid deflation produces a buckling instability that leads to the formation of epithelial folds. Buckling occurs across different length scales to overcome compressive stresses, and folding patterns become more complex with increasing size. I will discuss the potential of guiding the folds by controlling the shape and size of the epithelia.
\item Finally, in Chapter 9, I will summarize our findings and report our conclusions.
\end{itemize}

In summary, we present a microfluidic-based technique to impose a controlled deformation on an epithelial monolayer while continuously monitoring its state of stress. This technique allows us to investigate the active viscoelasticity of epithelial layers over physiological time scales. We also present a 3D model of the epithelium, which explains the observed phenomena using the active viscoelastic properties of the actomyosin cortex. Additionally, we demonstrate that these viscoelastic properties, along with adhesion micropatterning, can be utilized to engineer epithelial wrinkles with predictable geometry.

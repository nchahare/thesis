\hypertarget{objectives}{%
\section{Objectives}\label{objectives}}

\subsection*{General aim of the thesis} 

This thesis aims to investigate the mechanics of epithelial tissues under controlled pressure.  

\subsection*{Specific aims of the thesis} 
General aims are divided into specific goals:

\begin{enumerate}
\def\labelenumi{\arabic{enumi}.}
\item  Develop a novel technology for constructing three-dimensional epithelia using lumen pressure control.  
\item  Characterize the material response of the pressurized epithelial tissue.  
\item  Explore the mechanics of epithelial folds.
\end{enumerate}

\hypertarget{thesis-outline}{%
\section{Thesis outline}\label{thesis-outline}}

Results are presented in Part 2 with four chapters that address the specific aims of the thesis and provide an understanding of the mechanics of epithelial layers subjected to controlled pressure.  

\begin{itemize}
\item Chapter 6 details the construction of an experimental system designed to physically control epithelial monolayers. This chapter showcases the main result of the PhD, a novel microfluidic system that generates 3D epithelia with controlled pressure and shape. The chapter highlights the successful development of the microfluidic system, while also summarizing any failed or attempted methods used in constructing the device.  
\item Chapter 7 focuses on using the microfluidic device to understand epithelial mechanics. The chapter reports the results of rheological experiments and relates them to a computational framework that explains the observed phenomenology in terms of the viscoelasticity of the actomyosin cortex. 
\item Chapter 8 describes a buckling instability in pressurized epithelia. It is found that rapid deflation produces a buckling instability that leads to the formation of epithelial folds. Buckling occurs across different length scales to overcome compressive stresses, and folding patterns become more complex with increasing size. The chapter discusses the potential of guiding the folds by controlling the shape and size of the epithelia. 
\item Finally, in Chapter 9, the findings are summarized with a list of conclusions along with a brief discussion on future perspectives of this thesis.
\end{itemize}

In summary, the thesis presents a microfluidic-based technique to impose a controlled deformation on an epithelial monolayer while continuously monitoring its state of stress. This technique allows for investigation of the active viscoelasticity of epithelial layers over physiological time scales. The thesis also presents a 3D model of the epithelium, developed by Adam Ouzeri and Marino Arroyo, which explains the observed phenomena using the active viscoelastic properties of the actomyosin cortex. Furthermore, it is demonstrated that these viscoelastic properties, along with adhesion micropatterning, can be utilized to engineer epithelial wrinkles with predictable geometry. The results provide an understanding of the mechanics of epithelial layers subjected to controlled pressure and showcase the potential of the developed techniques to further explore the synthetic morphogenesis.

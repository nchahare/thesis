\hypertarget{what-is-to-be-done}{%
\section{What is to be done?}\label{what-is-to-be-done}}

Morphogenesis refers to the process of tissue deformation or growth, which results from the combination of both endogenous and exogenous mechanical forces \cite{valet2022, collinet2021}. These forces may arise from the contractility of the epithelium and the surrounding matrix, as well as hydraulic pressure from the lumen \cite{torres-sanchez2021, chan2020}. The various stresses act on different components of the tissue, such as cells and the extracellular matrix, which exhibit unique viscoelastic properties and remodeling time scales \cite{cavanaugh2020, kelkar2020, ambrosi2019}. However, comprehending how these stresses interact with viscoelastic properties to bring about particular morphogenetic events in vivo presents significant technical and conceptual challenges. These obstacles include disentangling the roles played by distinct components in a system, a lack of tools for quantitative measurements of stresses and mechanical properties, and an inability to apply controlled stresses over a wide range of amplitudes and rates.

In response to these challenges, bottom-up approaches have emerged as a complementary strategy for understanding the morphogenetic potential of individual components and building complex, functional tissues \cite{trentesaux2023, ingber2018}. These approaches have been successful in engineering basic morphogenetic processes such as epithelial bending or buckling \cite{matejcic2022}. However, even though bottom-up approaches are proving to be successful, we still need tools that can measure and control the shape and stress of 3D epithelia simultaneously. Additionally, we lack computational models that integrate cellular and tissue shape with the subcellular determinants of epithelial mechanics, such as the contractility, turnover, and viscoelasticity of the actomyosin cortex.

This thesis seeks to address these gaps in knowledge by investigating the mechanics of epithelial tissues. A comprehensive understanding of the principles that govern tissue form and function is essential for both advancing our understanding of fundamental physical rules in biology and inspiring new engineering tools and design principles. To achieve this, we leverage cutting-edge technologies, such as 3D printing, microfluidics, and 3D cell cultures, to individually control morphogenetic driving factors. 

Our approach provides a material science perspective for probing the intricate mechanisms involved in the generation of forces and shape changes at the cellular and tissue levels, and holds promise for discovering emergent phenomena and enabling the building of novel tissue forms and assemblies.  

\newpage

\hypertarget{objectives}{%
\section{Objectives}\label{objectives}}

\subsection*{General aim of the thesis} 

This thesis aims to investigate the mechanics of epithelial layers under controlled pressure.  

\subsection*{Specific aims of the thesis} 
General aims are divided into specific goals:

\begin{enumerate}
\def\labelenumi{\arabic{enumi}.}
\item  Develop a novel technology for constructing three-dimensional epithelia using lumen pressure control.  
\item  Characterize the material response of the pressurized epithelial tissue.  
\item  Explore the mechanics of epithelial folds.
\end{enumerate}

\hypertarget{thesis-outline}{%
\section{Thesis outline}\label{thesis-outline}}

Results are presented in Part 2 with four chapters that address the specific aims of the thesis and provide a understanding of the mechanics of epithelial layers subjected to controlled pressure.  

\begin{itemize}
\item Chapter 6 details the construction of an experimental system designed to physically control epithelial monolayers. This chapter showcases the main result of the PhD, a novel microfluidic system that generates 3D epithelia with controlled pressure and shape. The chapter highlights the successful development of the microfluidic system, while also summarizing any failed or attempted methods used in constructing the device.  
\item Chapter 7 focuses on using the microfluidic device to understand epithelial mechanics. The chapter reports the results of rheological experiments and relates them to a computational framework that explains the observed phenomenology in terms of the viscoelasticity of the actomyosin cortex. 
\item Chapter 8 describes a buckling instability in pressurized epithelia. It is found that rapid deflation produces a buckling instability that leads to the formation of epithelial folds. Buckling occurs across different length scales to overcome compressive stresses, and folding patterns become more complex with increasing size. The chapter discusses the potential of guiding the folds by controlling the shape and size of the epithelia. 
\item Finally, in Chapter 9, the findings are summarized with a list of conclusions along with a brief discussion on future perspectives of this thesis.
\end{itemize}

In summary, the thesis presents a microfluidic-based technique to impose a controlled deformation on an epithelial monolayer while continuously monitoring its state of stress. This technique allows for investigation of the active viscoelasticity of epithelial layers over physiological time scales. The thesis also presents a 3D model of the epithelium, developed by Adam Ouzeri and Marino Arroyo, which explains the observed phenomena using the active viscoelastic properties of the actomyosin cortex. Furthermore, it is demonstrated that these viscoelastic properties, along with adhesion micropatterning, can be utilized to engineer epithelial wrinkles with predictable geometry. The results provide an understanding of the mechanics of epithelial layers subjected to controlled pressure and showcase the potential of the developed techniques to further explore the synthetic morphogenesis.
